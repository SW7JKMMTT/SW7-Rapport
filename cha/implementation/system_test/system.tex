\chapter{System \& Load Testing}\label{cha:system_and_load_testing}
In this chapter we will describe how we perform a system and load test on our server;
then we will present results of said tests and conclude on them.
Lastly we will reflect on what these results mean going forward, and what they can do for us.

\bigskip
To test our server application as a whole, we use \textit{Apache JMeter}\footnote{\url{http://jmeter.apache.org/}} which is a open source application designed for testing web services among others.
This tools allows us to simulate a typical workflow of using our REST API, while gathering various data about the performance and result of specific requests to the server.

Because \textit{JMeter} can be used to simulate the workflow, we are testing both the performance and the functional behaviour of the server.
Meaning that this test also is a system test -- And by increasing the amount of requests we send to the server in any given test, we can shift focus towards load testing.

Moreover \textit{JMeter}, can also distribute tests such that multiple computers can be used to load test the server.
By utilising this feature we can assure that the machine executing the test is not the bottleneck, and the outcome of any test will reflect the performance of the server.

\section{Test Plans}
When doing system testing we focus on testing the interaction between components in the system.
For us, this means using the server through its REST API and trying to cover all, or at least the most significant, endpoints.
Because of this two test plans are made; one for submitting data to the server, and one for fetching data from the server.
We do not deem it necessary to include features such as uploading profile images, since these endpoints are insignificant for the business logic.

\subsection{Submitting Data}
Our first test plan aims to react all functional behaviour which is involved in a producer clients typical workflow.
The test plan can be summarized by the following points:
\begin{enumberate}
    \item Setup $n$ new test users.
    \begin{enumberate}
        \item Get authorization token for existing super user.
        \item Create $n$ new users and get authorization token for each.
        \item Change the given name of each user.
    \end{enumberate}
    \item Create $m$ vehicles and save their ids.
    \item Create $x$ routes with random vehicle and random user, and randomly select a center point for each route.
    \item For each route:
    \begin{enumberate}
        \item Mark route as \code{ACTIVE}
        \item In a loop, create $y$ waypoint and datapoint pairs in random order, with a minimum delay of 1 second between each pair.
              All waypoints on are placed within a given radius of the route's center point, and the datapoints consists of \textit{current speed} and \textit{fuel level}.
        \item Mark route as \code{COMPLETE}
    \end{enumberate}
\end{enumberate}

It should be noted that this test plan only submits data to the server, and advanced queries are not performed.
This means that the result only represents the server's ability to receive data.

\subsection{Fetching Data}
To analyse the performance of the server while executing queries, both simple and complex, and fetching data, we set up a test plan which does this.
In the test we want to gauge the performance of
\begin{enumberate*}
    \item regular fetching of data, e.g. getting all users;
    \item more complex fetching, e.g. getting all routes associated with a given driver; and
    \item geospatial queries, which we reckon will be the most demanding type of fetching data, performance wise.
\end{enumberate*}



\section{Results}
Some even nicer graphs here.

\subsection{NoSQL vs. SQL}
FIGURES MAN

\subsection{General Performance}
FIGURES MAN

\section{Effects}
We found this stuf
\begin{description}
    \item[High Response Times]\hfill \\
        First we sweeep SQL in. Then we index shit
    \item[Consistency Errors]\hfill \\
        Duplicate indeces in NoSQL. Duplicate driver.
    \item[Persistence Errors]\hfill \\
        Something something set route.
\end{description}
