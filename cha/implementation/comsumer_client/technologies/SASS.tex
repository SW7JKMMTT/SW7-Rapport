\subsection{SASS} \label{ssec:scss}
% Taget fra wikipedia!
% Sass is a scripting language that is interpreted into Cascading Style Sheets (CSS).
% SassScript is the scripting language itself.
% The newer syntax, "SCSS", uses block formatting like that of CSS.
% The indented syntax and SCSS files are traditionally given the extensions .sass and .scss, respectively.

% CSS3 consists of a series of selectors and pseudo-selectors that group rules that apply to them.
% Sass (in the larger context of both syntaxes) extends CSS by providing several mechanisms available in more traditional programming languages, particularly object-oriented languages, but that are not available to CSS3 itself.
% When SassScript is interpreted, it creates blocks of CSS rules for various selectors as defined by the Sass file.
% The Sass interpreter translates SassScript into CSS.
% Alternately, Sass can monitor the .sass or .scss file and translate it to an output .css file whenever the .sass or .scss file is saved.
% Sass is simply syntactic sugar for CSS.

% The official implementation of Sass is open-source and coded in Ruby; however, other implementations exist, including PHP, and a high-performance implementation in C called libSass.
% There's also a Java implementation called JSass.
