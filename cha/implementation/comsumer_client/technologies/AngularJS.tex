\subsection{AngularJS} \label{ssec:angular}
AngularJS is a front-end framework developed and maintained by Google, which generates a single-page web application.
It is open-source and uses components written in Javascript to generate its content.
One of the AngularJS goals of AngularJS is to ease the development and testing of such a web application, this is done by producing a Javascript framework which assists the MVC and the MVVM design patterns.

When compiling the AngularJS project, the compiler firstly reads through the HTML page looking for embedded tags with custom attributes, which is interpreted.
Then these attributes are used to directly bind data from the defined components into those HTML tags, thereby representing the defined variables in the HTML view.
AngularJS is an consists of a MEAN stack, which means it consists of a database which is MongoDB, a web application which is Express.js, AngularJS it self, and a runtime environment which is Node.js.

The framework it self extends the HTML in order for it to present the bound data through two-way data-bindings which allows it to make automatic synchronization between the models and views.
This makes it possible for AngularJS to decrease the amount of explicit DOM manipulation which allows for improved performance.

According to AngularJS some of the design goals is to decentralize the manipulation of the HTML files from the applications logic it self, and for the client side and the server side to be deployed separately.
