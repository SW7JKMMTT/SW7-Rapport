\section{System \& Load Testing}\label{cha:system_and_load_testing}
In this chapter we describe how we perform system and load testing on our server;
then we present results of said tests and conclude upon them.
Lastly we reflect on what these results mean going forward, and what they can do for us.

Throughout this chapter we comment on and evaluate the \textit{performance} of the server.
We define \textit{performance} of the server as a measure for both response time of requests and how many requests per second the server can handle.
To gauge performance we also use the term \textit{load}, which we define as follows;
if a system is \textit{under heavy load} it means that a significantly larger number of users are using it, compared to \textit{under normal load}, which is when a realistic number of users are using the system.
When we use the term \textit{under load} without any quantifier, we mean usage such that the system is not idle.

\bigskip
Because scalability is a significant non--functional requirement for our system, as described in \cref{sec:requirements}, we want to gauge the load scalability of our system, such that we can identify what kind of load a single instance of our server can handle --- this is done with load testing. % chktex 08
It would be possible to configure a second instance of the server and use load balancer to distribute traffic, i.e.~horizontal scaling, but due to time constraints this test remains undone.

The system testing is used to both determine faults in the system's functionality, and to ensure that the implementation of the specification for API usage is correct.

\bigskip
To test our server application as a whole, we use \textbf{Apache JMeter}\footnote{\url{http://jmeter.apache.org/}} which is an open source application designed for testing web services and other services.
This tool allows us to simulate a typical workflow of our REST API, while gathering various data about the performance and results of specific requests to the server.

Because JMeter can be used to simulate the workflow, we are testing both the performance and the functional behaviour of the server.
Meaning that this test is also a system test --- and by increasing the amount of requests we send to the server in any given test, we can shift focus towards load testing.

Moreover, JMeter can also distribute tests such that multiple computers can be used to load test the server.
By utilising this feature we can assure that the machine executing the test is not the bottleneck, and the outcome of any test will reflect the performance of the server.

\bigskip
To better understand the results of our system and load tests, we use various profiling tools on the server.
These tools are capable of measuring which parts of the system are most critical when under load, and can therefore give an insight in where changes potentially can increase performance.
Additionally we monitor server logs and use debugging tools to identify any faults which may surface during testing.
