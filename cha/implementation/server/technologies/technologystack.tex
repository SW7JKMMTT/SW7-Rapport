% chktex-file 08 chktex-file 13 chktex-file 17
\section{Technology Stack}
To start with a good foundation for our server, it is important that we choose an appropriate stack of technologies, such that we take advantage of well tested and supported frameworks and libraries.
This approach enables us, as developers, to focus on parts unique to the system we are developing, i.e.~business logic.
Moreover since the server is what connects to both ends of the system, i.e.~the producer clients, and the front--end clients, it is paramount that the outwards facing interfaces on the server are stable, thoroughly tested, and well documented.

\subsection{The Programming Language}
It is important for us to choose a programming language, in which we all are comfortable, and which will not be a hindrance for the system.
This means that the programming language and the environment around it must allow for; scalability, reliable and good performance, as well as the ability to utilise libraries for web related tasks, e.g.~running a server and routing/handling incoming requests.

Furthermore it is important that the development environment is available for multiple platforms, such that all group members can participate equally, since we use both macOS, Linux, and Windows.

\bigskip
To decide which language to use, we look at popular languages fitting the above mentioned requirement;
these are Node.js\footnote{\url{https://nodejs.org/en/}} and Python with Django\footnote{\url{https://www.djangoproject.com/}} or Flask\footnote{\url{http://flask.pocoo.org/}}, which are both popular for developing web services.
However both of these are dynamically typed languages, which can be good for rapid development but bad for the maintainability of a system\cite{kleinschmager2012static}.
Moreover especially Node.js is not as mature and lack the same enterprise libraries and safeties which come with yet another language --- Java\footnote{\url{https://www.java.com/}}.

Java is a mature and well documented language that have been tried and tested in numerous enterprise systems; and since 2001 Java has resided in the top two, but mostly top one, languages on the TIOBE index\cite{TiobeIndex}.
It also enables us to develop a web service, by using a combination of the many high quality third party libraries built for such tasks.
Moreover, Java is platform independent both in development and in production, so we are ensured that all group members can get involved in the development.
Because of this, and because it also fits the above mentioned requirements, we choose to develop the server of our system in Java.

\subsection{Java 8}
After having chosen Java as the programming language we use to develop our server, the next step is to ensure that we use the concepts and libraries necessary to develop efficiently.
In our case, this means that we can develop individual components which can be reused and replaced if new or other functionality is needed --- in other words, we want to build a system with low coupling and high cohesion.

In the Java world, you can use JavaEE\footnote{\url{http://www.oracle.com/technetwork/java/javaee/overview/index.html}}, where EE stands for ``Enterprise Edition''.
It contains a large number of frameworks and libraries for building large and complex enterprise applications.
One of the compelling features of JavaEE is its implementation of easy context and dependency injection, which if used properly can ensure low coupling.

However, as a part of the enterprise robustness of JavaEE, it is always a version behind the main Java language.
This means that some of the new features of Java 8, such as lambda expressions in streams and switch cases over strings, are not available if we choose JavaEE.

Because of this, and because of some added features and overall improvements to the ease of use, we instead choose to work with Spring\footnote{\url{https://spring.io/}}.
Spring is a framework on top of Java, which enables the use of features similar to those in JavaEE.
Moreover many other third party frameworks and libraries have great integration with Spring, which will ease the initial setup of the different technologies.

\subsection{Communication}
As the server in our system needs to communicate with other parts of the system over the internet, we need a unified way of passing information around.
The two de facto standards for this type of communication are SOAP and REST.

We present a short summary of the two technologies:
\begin{description}
    \item[SOAP] (Simple Object Access Protocol)\cite{SOAP_spec}\hfill\\
        By passing XML data over HTTP (and in some cases HTTPS or SMTP), SOAP enables computers to communicate.
        SOAP requires a specific packaging of data in a so called \textit{SOAP Envelope} which contains a header and a body.
        Mainly SOAP focusses on making procedures available to other systems via the internet.
    \item[REST] (Representational State Transfer)\cite{RESTful_best_practices}\hfill\\
        Services implementing a REST interface are often described as RESTful services.
        Instead of enforcing a predefined data structure as SOAP does, REST utilises the HTTP methods, e.g.~GET, POST, PUT, and DELETE.
        Moreover requests to and responses from a RESTful service can be in any format, but often XML or JSON is used,
        and where SOAP serves procedures, REST focusses on serving resources.

        State is not conserved on the server, which means that RESTful services are easily scalable --- two sequential requests may be on two different servers.
        Furthermore the flow through the application is driven by the requests the user makes, and any state is only kept on the client.
\end{description}

We choose to implement the server as a RESTful service, because we deem that it fits our needs the most.
Additionally all group members have experience with developing RESTful services, and the available libraries for developing a RESTful service in Java are well documented and mature.

\bigskip
In the Java world there exists a concept called Javax which embodies various extensions to the Java standard.
One component of Javax is the JAX--RS specification\footnote{\url{https://jax-rs-spec.java.net/}}, which describes a RESTful service in Java.
Our RESTful server will be based upon the RESTEasy\footnote{\url{http://resteasy.jboss.org/}} implementation of version 2.0 of the aforementioned JAX--RS specification.

RESTEasy is developed by JBoss, owned by Red Hat, and is a well documented set of frameworks for integrating REST into a Java server application or service.\fxnote{Måske tilføj noget med hvordan vi bruger det, dog i et senere afsnit!}
It allows us to easily develop a RESTful service while being closely integrated into Spring.
Moreover RESTEasy handles the HTTP communication, and does it asynchronously, which can greatly improve performance and scalability,
since multiple inbound requests can be handled simultaneously.

\subsection{Deployment}
To be able to deliver our service we need what is called an application server, i.e.~something which can run our code and serve it to the outside world in a production environment.

Due to our earlier choice of RESTEasy, we will be using the WildFly\footnote{\url{http://wildfly.org/}} application server, because it is also developed by JBoss.
This ensures that our different technologies can work in conjunction right from the start;
which in turn means that we can implement a working prototype, and focus on our business logic instead of tedious configuration of the systems underlying technologies.

\subsection{Data Persistence}\label{subsec:datapersistence}
Lastly we need to decide on a way of persisting data, i.e.~storing the data we gather.
In choosing the data persistence technologies these three criteria are the most significant for a database system:
\begin{eletterate}
    \item High performance storage and retrieval of data.
    \item Making geospatial queries, i.e.~queries which finds data with locational delimiters.
    \item Solid and supported integration with Java.
\end{eletterate}


\bigskip
There exists numerous categories of database systems or paradigms, but the most prevalent are SQL and NoSQL.
Furthermore there exists NewSQL database systems, which are relational but with the scalability of NoSQL;
however, as this is a relatively new concept the support and integration into our Java environment is lackluster, so we will ignore these specific NewSQL database systems.

\subsubsection*{SQL}

SQL database systems are relational and data is stored in a number of tables, some examples of popular SQL database systems are MySQL\footnote{\url{https://www.mysql.com/}}, PostgreSQL\footnote{\url{https://www.postgresql.org/}}, and IBM DB2\footnote{\url{http://www.ibm.com/analytics/us/en/technology/db2/}}.\cite{DB_RANKINGS}

A key property of SQL database systems is that they use transactions which are ACID--compliant\cite{Haerder:1983:PTD:289.291}, making them ideal for use in environments where it is critical transactions only have expected or recoverable effects, e.g.~the financial world.

SQL database systems are the primary choice for commercial use\cite{dbs_concepts}.
This is supported with the fact that 7 out of the top 10 most popular database systems in December 2016 are SQL database systems\cite{DB_RANKINGS}.
Popularity is an important factor when discussing database systems, since it shows how much the systems are used, their maturity, and how tested in production the systems are.
A widely used database system gives a indicator of that it is reliable and have good integration options,
since unreliable systems with bad integration options will not be very popular. 

\subsubsection*{NoSQL}
NoSQL database systems is a term used to describe a myriad of different database back--ends.
Some of these are; key--value stores, document stores, and graph based systems.
Common for the majority of NoSQL database systems is that they value simplicity of design, scalability and speed over consistency.
This means that one significant shortcoming of NoSQL database systems is that they do not use ACID--compliant transactions.
However many NoSQL database systems are ACID--like, i.e.~read and write operations can be configured to show consistency.

Document stores are an expansion to Key--Value based stores, where the value is replaced by a document of some sorts, e.g.~XML or JSON formatted documents.
An advantage of document stores is that any two documents do not need to follow the same ``schema'', i.e.~documents can have a different amount and kind of properties.
This makes expanding the system significantly easier, since adding a new property, e.g.~fuel consumption, is as easy as introducing it into the server application.

\bigskip
As mentioned earlier, we want to use a solid and widely used database system, and therefore we choose to use a SQL database.

\subsubsection*{Interfacing With the Database System} % AKA Data Persistence Provider

The database system we choose must be compilable with Java.
There are several systems which can act as an interface between the program and the database system.
We would prefer a mature and well tested system, which is also relatively easy to operate, while having full control for performance and customization.
We looked at Spring Data JPA\footnote{\url{https://spring.io/}}, jOOQ\footnote{\url{http://www.jOOQ.org/}}, and Hibernate ORM\footnote{\url{http://hibernate.org/orm/}}.
All three of them implement the \ac{JPA}--standard\cite{JavaPersistenceAPI}, which is an \ac{API} that describes the management of data. 

Spring Data JPA is a layer on top of Hibernate ORM which provides some abstraction, and can auto--generate some queries, however we would have less control over how the persistence would occur, and this can be to our disadvantage.
jOOQ is an attempt to simplify persistence in Java, it has most of the same features as Hibernate ORM.
jOOQ points to Hibernate ORM as the de--facto standard in the Java ecosystem\cite{JOOQvsHIBERNATE},
Hibernate ORM is also a mature and stable system which is well supported. 

We therefore choose Hibernate ORM to act as a Data Persistence Provider. 

\subsubsection*{Selecting a Database System}

The top four popular SQL database systems are\cite{DB_RANKINGS}: 
\begin{enumerate}
\item Oracle\footnote{\url{https://www.oracle.com/database/index.html}}
\item MySQL\footnote{\url{https://www.mysql.com/}}
\item Microsoft SQL Server\footnote{\url{https://www.microsoft.com/da-dk/server-cloud/products/sql-server/overview.aspx}}
\item PostgreSQL\footnote{\url{https://www.postgresql.org/}}
\end{enumerate}

Oracle and Microsoft SQL Server will not be considered since Oracle is not free to use\cite{oracle_pricing}, 
and Microsoft SQL Server on Linux will first be available mid-2017\cite{ms_sql_linux}.

MySQL and PostgreSQL both have support for geospatial queries and both supports Hibernate ORM\cite{hibernate_support}, so to choose we look at their performance.
Regarding performance PostgreSQL have a slight advantage in simple database reads and a big advantage in more complex read queries.
MySQL has a slight advantage in database writes, 
so in all PostgreSQL is better in read queries and MySQL is only slightly better in writing.\cite{post_vs_mysql}

From the fact that PostgreSQL is a popular SQL database system, it supports geospatial queries and is well--performing,
is it chosen to use as the database system for our data persistence.

PostgreSQL is a popular SQL database system which is well--performing.
It has support for Hibernate ORM and geospatial queries.
Therefore we choose to is for our data persistence.
