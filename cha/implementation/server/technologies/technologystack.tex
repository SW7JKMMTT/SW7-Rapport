% chktex-file 08 chktex-file 13 chktex-file 17
\section{Technology Stack}
To start with a good foundation for our server, it is important that we choose an appropriate stack of technologies, such that we take advantage of well tested and supported frameworks and libraries.
This approach enables us, as developers, to focus on parts unique to the system we are developing, i.e.~business logic.
Moreover since the server is what connects to both ends of the system, i.e.~the producer clients, and the front--end clients, it is paramount that the outwards facing interfaces on the server are stable, thoroughly tested, and well documented.

\subsection{The Programming Language}
It is important for us to choose a programming language, in which we all are comfortable, and which will not be a hindrance for the system.
This means that the programming language and the environment around it must allow for; scalability, reliable and good performance, as well as the ability to utilise libraries for web related tasks, e.g.~running a server and routing/handling incoming requests.

Furthermore it is important that the development environment is available for multiple platforms, such that all group members can participate equally, since we use both macOS, Linux, and Windows.

\bigskip
To decide which language to use, we look at popular languages fitting the above mentioned requirement;
these are Node.js\footnote{\url{https://nodejs.org/en/}} and Python with Django\footnote{\url{https://www.djangoproject.com/}} or Flask\footnote{\url{http://flask.pocoo.org/}}, which are both popular for developing web services.
However both of these are dynamically typed languages, which can be good for rapid development but bad for the maintainability of a system\cite{kleinschmager2012static}.
Moreover especially Node.js is not as mature and lack the same enterprise libraries and safeties which come with yet another language --- Java\footnote{\url{https://www.java.com/}}.

Java is a mature and well documented language that have been tried and tested in numerous enterprise systems; and since 2001 Java has resided in the top two, but mostly top one, languages on the TIOBE index\cite{TiobeIndex}.
It also enables us to develop a web service, by using a combination of the many high quality third party libraries built for such tasks.
Moreover, Java is platform independent both in development and in production, so we are ensured that all group members can get involved in the development.
Because of this, and because it also fits the above mentioned requirements, we choose to develop the server of our system in Java.

\subsection{Java 8}
After having chosen Java as the programming language we use to develop our server, the next step is to ensure that we use the concepts and libraries necessary to develop efficiently.
In our case, this means that we can develop individual components which can be reused and replaced if new or other functionality is needed --- in other words, we want to build a system with low coupling and high cohesion.

In the Java world, you can use JavaEE\footnote{\url{http://www.oracle.com/technetwork/java/javaee/overview/index.html}}, where EE stands for ``Enterprise Edition''.
It contains a large number of frameworks and libraries for building large and complex enterprise applications.
One of the compelling features of JavaEE is its implementation of easy context and dependency injection, which if used properly can ensure low coupling.

However, as a part of the enterprise robustness of JavaEE, it is always a version behind the main Java language.
This means that some of the new features of Java 8, such as lambda expressions in streams and switch cases over strings, are not available if we choose JavaEE.

Because of this, and because of some added features and overall improvements to the ease of use, we instead choose to work with Spring\footnote{\url{https://spring.io/}}.
Spring is a framework on top of Java, which enables the use of features similar to those in JavaEE.
Moreover many other third party frameworks and libraries have great integration with Spring, which will ease the initial setup of the different technologies.

\subsubsection{POJOs \& Beans}
A concept which is familiar to any enterprise Java programmer is that of \ac{POJO}.
A \acp{POJO} is a Java object that follows no framework convention, such as implementing a specific interface, or extending framework classes.
Objects that are not \acp{POJO} are said to be \acp{SJO}.
\acp{POJO} lower coupling, by lowering the dependency on the underlying framework, and moving the responsibility of wiring the services from the usercode to the framework.
Typically implementing a \ac{POJO} means annotating the various methods, fields, and classes with certain Java annotations, which the framework interprets to figure out what it is supposed to wire up to the class.\cite{spring_pojo}

A Bean is a type of \ac{POJO}, but with additional restrictions.
Beans have to be serializable, implement getters and setters with the property name prefixed with either \code{get} or \code{set}, respectively, and have a no--arg constructor.
Because of the restrictions, a framework like Spring can instantiate the class and set the properties without any knowledge of definition.
Requiring a no--arg constructor does impose an issue.
Namely, it means that the object can be instantiated in an invalid state, which requires the programmer to knows which properties he must set to configure the object.
Strictly speaking a Bean is not a \ac{POJO}, since it has to implement the serializable interface, but since the interface has no methods, the Bean is still said to be a \ac{POJO}.
Spring allows for the no--arg constructor of a Bean to have an access level stricter than \code{public}.
We will use that to hide the constructor when its use would result in a wrongly configured object.

\subsubsection{Spring IoC Container}
Large projects, in object oriented languages, often have dependencies between objects.
The naive approach to resolve dependencies is to let an object instantiate, or find, its own dependencies.
Often called \textit{hardcoded dependencies}, they have a significant issue.
The class which declares the dependency has also decided what object will fulfil the dependency.
With the rise of mock based unit testing, where an instance of an object is swapped out for a mock object, hardcoding those dependencies is not possible, since it prevents mocking.

The solution often employed is the dependency injection pattern, which relies on letting the object doing the instantiation, resolve the dependencies.
By doing so, the calling object, instead of the callee, retains control over the dependencies, and can inject different dependencies for different usecases, such as mock objects for unit testing.

The Spring \ac{IoC} container takes the concept a step further.
Instead of the caller retaining control, the \ac{IoC} container controls all dependencies.
It instantiates and configures them according to metadata specified with either XML, code, or annotations.
Spring heavily relies on the concept of Beans described above, but also includes support for managing \acp{POJO} as if they were Beans.
Although the main configuration in Spring is done with XML, allowing the programmer to instantiate object into the \code{ApplicationContext}, it is also possible to get dependencies injected by using the \code{Autowired} annotation.
When Spring instantiates a Bean it first looks though it and injects dependencies into all the fields having the \code{Autowired} annotation, by resolving the dependency class/interface.
If the application includes advanced Beans, that need some form of lifetime management, Spring supports factory Beans, which are factories that create a specific type of Bean.

By inverting control to the container, instead of the objects, Spring makes it possible to switch out a large part of an application, by only changing XML.
We will be relying heavily on the Spring \ac{IoC} container in the server.

\subsubsection{Communication}
\label{subsec:tech_communication}
As the server in our system needs to communicate with other parts of the system over the Internet, we need a unified way of passing information around.
The two de facto standards for this type of communication are SOAP and REST.

We present a short summary of the two technologies:
\begin{description}
    \item[SOAP] (Simple Object Access Protocol)\cite{SOAP_spec}\hfill\\
        By passing XML data over HTTP (and in some cases HTTPS or SMTP), SOAP enables computers to communicate.
        SOAP requires a specific packaging of data in a so called \textit{SOAP Envelope} which contains a header and a body.
        Mainly SOAP focusses on making procedures available to other systems via the Internet.
    \item[REST] (Representational State Transfer)\cite{RESTful_best_practices}\hfill\\
        Services implementing a REST interface are often described as RESTful services.
        Instead of enforcing a predefined data structure as SOAP does, REST utilises the HTTP methods, e.g.~GET, POST, PUT, and DELETE.
        Moreover requests to and responses from a RESTful service can be in any format, but often XML or JSON is used,
        and where SOAP serves procedures, REST focusses on serving resources.

        State is not conserved on the server, which means that RESTful services are easily scalable --- two sequential requests may be on two different servers.
        Furthermore the flow through the application is driven by the requests the user makes, and any state is only kept on the client.
\end{description}

We choose to implement the server as a RESTful service, because we deem that it fits our needs the most.
Additionally all group members have experience with developing RESTful services, and the available libraries for developing a RESTful service in Java are well documented and mature.

\bigskip
In the Java world there exists a concept called Javax which embodies various extensions to the Java standard.
One component of Javax is the JAX--RS specification\footnote{\url{https://jax-rs-spec.java.net/}}, which describes a RESTful service in Java.
Our RESTful server will be based upon the RESTEasy\footnote{\url{http://resteasy.jboss.org/}} implementation of version 2.0 of the aforementioned JAX--RS specification.

RESTEasy is developed by JBoss, owned by Red Hat, and is a well documented set of frameworks for integrating REST into a Java server application or service.\fxnote{Måske tilføj noget med hvordan vi bruger det, dog i et senere afsnit!}
It allows us to easily develop a RESTful service while being closely integrated into Spring.
Moreover RESTEasy handles the HTTP communication, and does it asynchronously, which can greatly improve performance and scalability,
since multiple inbound requests can be handled simultaneously.

\subsection{Serialization}
As can be read in \cref{subsec:tech_communication} we decided to use REST as a general design guideline for the server.
Since the REST standard permits any form of representation of the data, we have to pick a format to carry our data.
We chose JSON, since it has a lower size overhead than XML, and is a newer standard which has gained massive traction in later years, but XML would be suitable as well.
Although we have picked JSON, it would be possible to have an XML representation of the API running using mostly the same codebase.

To serialize our object into JSON from \acp{POJO} we have decided to use Jackson\footnote{\url{https://github.com/FasterXML/jackson}}, which is a library for serializing \acp{POJO} to JSON or XML, with JAX-RS support.
Jackson is, like a lot of the other \ac{Java EE} libraries, configured using annotations.
Among the most important annotations are \code{@JsonIgnore}, which causes a property to be ignored, and \code{@JsonSetter/Getter}, which can be used in conjunction with \code{@JsonIgnore} to create write or read only properties.
Another feature that integrates with the rest of the technology stack is the integration with JAX-RS, which means that the various end--point handlers, or methods, can return a Java object, and let the JAX-RS framework handle the serialization.

\subsection{Deployment}
To be able to deliver our service we need what is called an application server, i.e.~something which can run our code and serve it to the outside world in a production environment.

Due to our earlier choice of RESTEasy, we will be using the WildFly\footnote{\url{http://wildfly.org/}} application server, because it is also developed by JBoss.
This ensures that our different technologies can work in conjunction right from the start;
which in turn means that we can implement a working prototype, and focus on our business logic instead of tedious configuration of the system's underlying technologies.

\subsection{Data Persistence}\label{subsec:datapersistence}
Lastly we need to decide on a way of persisting data, i.e.~storing the data we gather.
In choosing the data persistence technologies these three criteria are the most significant for a database system:
\begin{eletterate}
    \item High performance storage and retrieval of data.
    \item Making geospatial queries, i.e.~queries which finds data with locational delimiters.
    \item Solid and supported integration with Java.
\end{eletterate}


\bigskip
There exists numerous categories of database systems or paradigms, but the most prevalent are SQL and NoSQL.
Furthermore there exists NewSQL database systems, which are relational but with the scalability of NoSQL;
however, as this is a relatively new concept the support and integration into our Java environment is lackluster, so we will ignore these specific NewSQL database systems.

\subsubsection*{SQL}

SQL database systems are relational and data is stored in a number of tables, some examples of popular SQL database systems are MySQL\footnote{\url{https://www.mysql.com/}}, PostgreSQL\footnote{\url{https://www.postgresql.org/}}, and IBM DB2\footnote{\url{http://www.ibm.com/analytics/us/en/technology/db2/}}.\cite{DB_RANKINGS}

A key property of SQL database systems is that they use transactions which are ACID--compliant\cite{Haerder:1983:PTD:289.291}, making them ideal for use in environments where it is critical transactions only have expected or recoverable effects, e.g.~the financial world.
A critical transaction is a transaction which must not go wrong and leave the database in an inconsistent state, for instance the transfer of money.

SQL database systems are the primary choice for commercial use\cite{dbs_concepts}.
This is supported with the fact that 7 out of the top 10 most popular database systems in December 2016 are SQL database systems\cite{DB_RANKINGS}.
Popularity is an important factor when discussing database systems, since it shows how much the systems are used, their maturity, and how tested in production the systems are.
A widely used database system gives a indicator of that it is reliable and have good integration options,
since unreliable systems with bad integration options will not be very popular. 

\subsubsection*{NoSQL}
NoSQL database systems is a term used to describe a myriad of different database back--ends.
Some of these are; key--value stores, document stores, and graph based systems.
Common for the majority of NoSQL database systems is that they value simplicity of design, scalability and speed over consistency.
This means that one significant shortcoming of NoSQL database systems is that they do not use ACID--compliant transactions.
However many NoSQL database systems are ACID--like, i.e.~read and write operations can be configured to show consistency.

Document stores are an expansion to key--value based stores, where the value is replaced by a document of some sorts, e.g.~XML or JSON formatted documents.
An advantage of document stores is that any two documents do not need to follow the same ``schema'', i.e.~documents can have a different amount and kind of properties.
This makes expanding the system significantly easier, since adding a new property, e.g.~fuel consumption, is as easy as introducing it into the server application.

\bigskip
As mentioned earlier, we want to use a solid and widely used database system, and therefore we choose to use a SQL database.

\subsubsection*{Interfacing With the Database System} % AKA Data Persistence Provider

The database system we choose must be compilable with Java.
There are several systems which can act as an interface between the program and the database system.
We would prefer a mature and well tested system, which is also relatively easy to operate, while having full control for performance and customization.
We looked at Spring Data JPA\footnote{\url{https://spring.io/}}, jOOQ\footnote{\url{http://www.jOOQ.org/}}, and Hibernate ORM\footnote{\url{http://hibernate.org/orm/}}.
All three of them implement the \ac{JPA}--standard\cite{JavaPersistenceAPI}, which is an \ac{API} that describes the management of data. 

Spring Data JPA is a layer on top of Hibernate ORM which provides some abstraction, and can auto--generate some queries, however we would have less control over how the persistence would occur, and this can be to our disadvantage.
jOOQ is an attempt to simplify persistence in Java, it has most of the same features as Hibernate ORM.
jOOQ points to Hibernate ORM as the de--facto standard in the Java ecosystem\cite{JOOQvsHIBERNATE},
Hibernate ORM is also a mature and stable system which is well supported. 

We therefore choose Hibernate ORM to act as a Data Persistence Provider. 

\subsubsection*{Selecting a Database System}

The top four popular SQL database systems are\cite{DB_RANKINGS}: 
\begin{enumerate}
\item Oracle\footnote{\url{https://www.oracle.com/database/index.html}}
\item MySQL\footnote{\url{https://www.mysql.com/}}
\item Microsoft SQL Server\footnote{\url{https://www.microsoft.com/da-dk/server-cloud/products/sql-server/overview.aspx}}
\item PostgreSQL\footnote{\url{https://www.postgresql.org/}}
\end{enumerate}

Oracle and Microsoft SQL Server will not be considered since Oracle is not free to use\cite{oracle_pricing}, 
and Microsoft SQL Server on Linux will first be available mid--2017\cite{ms_sql_linux}.

MySQL and PostgreSQL both have support for geospatial queries and both supports Hibernate ORM\cite{hibernate_support}, so to choose we look at their performance.
Regarding performance PostgreSQL have a slight advantage in simple database reads and a big advantage in more complex read queries.
MySQL has a slight advantage in database writes, 
so in all PostgreSQL is better in read queries and MySQL is only slightly better in writing.\cite{post_vs_mysql}

PostgreSQL is a popular SQL database system which is well--performing.
It has support for Hibernate ORM and geospatial queries.
Therefore we choose it for our data persistence.

\subsubsection{Data Persistence Provider}
Hibernate is another project operated by Red Hat.
It was originally just an \ac{ORM}, but has since expanded into Hibernate \ac{OGM}, which brings the power of Hibernate to NoSQL databases, and other projects.
Hibernate \ac{ORM} has been around for a long time and is according to themselves popular and used by ``tens of thousands of Java developers''.\cite{hibernate_pop}
This means that the Hibernate interface, and engine, is much more mature and stable than the \ac{OGM} implementation.
Hibernate \ac{ORM} also allows us to change the underlying database easily, which becomes important when we want to run integration tests, without configuring a full database on every development machine.
Since \ac{OGM} aims to be a drop in replacement, Hibernate allows developers to switch between SQL and NoSQL databases with relative ease.

The power of Hibernate comes from its ability to persist \acp{POJO} transparently.
Hibernate exposes its power via either its native Hibernate API, or a more generic \ac{JPA} implementation.
The native Hibernate interface has the advantage of being specific to Hibernate, and therefore it provides more direct and detailed access to the Hibernate persistence engine.
That is also the biggest shortcoming of the native interface.
Being specific to Hibernate means that an application using it is deeply dependent on the way Hibernate functions, and its specific implementations.
Enter the \ac{JPA}, which provides a generic, high level, and well supported interface for persisting objects, mainly based around annotations of the persistence \code{entities}.
Hibernate has support for the \ac{JPA}, and even allows mixing it with the native API, allowing developers to mix in specific Hibernate features when needed.
An additional nicety of using the \ac{JPA} is that Spring has support for configuring \ac{JPA} providers, like Hibernate, through the normal \code{Bean Factory}.

The \ac{JPA} works with \code{entity managers}.
It is through the \code{entity manager} that an application interfaces with the database, whether that be through a query to get data or, or a persist call to persist an entity.
The update of entities is done transparently by using Java proxies, which injects information about entity changes into the persistence providers, which persists them on transaction close.
When interfacing with Spring specifically, the \code{entity manager} is injected into a transactional Bean, where Spring will manage the transactions.
The transactions can be managed manually, but letting Spring do it allows the container \fxnote{Write about the Spring IoC container} to figure out when the outermost transaction ends, and therefore minimize the amount of transactions.
We use the \ac{JPA}, since it provides all the features we need, while also being portable between different persistence providers.

\subsubsection{Spatial Querying}
Since we, as described in \cref{sec:requirements}, want to keep track of vehicle positions, we will be storing vehicle location, which is geospatial data.
To use the data, we need some way of querying the data, and using it in processing or returning it to the client.
Spatial data is different from regular data, in that the way you want to query often has to do with distance in a 2--D space, 3--D globe, or points in relation to a polygon.

PostgreSQL, our chosen database system, has an available extension, called PostGIS\footnote{\url{http://www.postgis.net/}}, which adds support for spatial objects/queries.
PostGIS is a powerful extension, but Hibernate has no support for it.
The \ac{JPA} does allow for native \ac{SQL} queries to be executed in the underlying databases native dialect, which we could use to run native spatial queries on the PostgreSQL/PostGIS database.
Since we have to write the \ac{DDL} for the database in hand, we could trivially add the required constraints for PostGIS to function.
An issue with such a solution however is that it would tie us to PostgreSQL as our database, and negate a lot of the modularity that Hibernate affords us.

An alternative, that does have support in Hibernate, is Hibernate Search, another Hibernate project, specifically for full text search and geospatial querying.
Hibernate Search uses Lucene\footnote{\url{https://lucene.apache.org/}} to index entites with Hibernate Search specific annotations, and allows you to query for them using either Hibernate Search, or Lucene queries.
To make integration easy, Hibernate Search allows the developer to get a Hibernate Search Enhanced \code{full text entity manager} from a regular \code{entity manager}.
Since Hibernate Search requires almost no extra configuration to get going, it is easy to integrate into a project which already uses Hibernate and the \ac{JPA}.
When querying for data Hibernate Search returns Hibernate managed entites, making a switch from Hibernate Search to Hibernate querying practically transparent.
By using Lucene, Hibernate Search is practically database agnostic, meaning that it maintains the modularity that Hibernate provides to a project.
To scale the server in the future, Hibernate Search also allows us to cluster the indexing.
We have chosen Hibernate Search as it is practically transparent, but also has the power we need.
The ability to scale out as well as up was also an important factor.


\subsubsection{Flyway}
Flyway is a lightweight database migration tool.
Meaning that it applies database migration scripts in sequence, remembering the current point in the sequence between runs.
That relatively simple functionality is hugely useful for incrementally keeping a database up to date with a changing codebase.
Since a bare bones migration tool only keeps track of where it currently is, it is crucial that the old scripts do not change, as that would give an inconsistent version of the database if the scripts were reapplied.
Flyway helps guard against such problem by storing a checksum for the script, which it checks for every deployment, that way it is immediately obvious if an already deployed script has been changed.
Should a migration script fail Flyway will stop the migration and fail the deployment, prompting the developer to fix the problem manually.
Such a problem should be avoided, since it takes the production server down until the problem is fixed.

The migration scripts are found at preconfigured paths in the classpath of the Java archive.
Upon deployment Flyway will scan the directories for files in the format \code{V(version)\_\_(description).sql}, which it will apply to the database.
The version in front determines in what order the scripts will be applied.
Once again it is of utmost importance that the list of scripts is only appended to, if a script is inserted anywhere but the end, Flyway will fail the deployment, since the state of the database can not be guaranteed.

A migration script, for us, is an SQL script, which will migrate the \ac{DDL} and the data forwards, to a new schema.
It is important that the data inside the database is migrated correctly to the new \ac{DDL}, since otherwise customers might lose their data.
