% vim: spell:tw=80
\section{Authentication}

Since the system will store proprietary business data, we need some form of
access control. Both to keep unwanted users out, but also to keep people with
access to one part of the system, from modifying data in another part.

Authentication in a REST API can in principle be done in four ways:

\begin{enumberate}
    \item Using a custom HTTP header.
    \item Using HTTP basic authorization.
    \item Using a custom HTTP authorization scheme.
    \item Using a query parameter.
\end{enumberate}

We wanted to follow the HTTP specification as closely as possible, to stay in
line with the spirit of REST\@. While HTTP does allow for custom headers, it
already has an accepted method of doing authentication. Since we want use HTTP
accepted standards where applicable this rules out methods 1 and 4, which are
both ad--hoc solutions to an already solved problem. Picking between 2 and
3 requires looking at the problem in context. Since the main use of our service
is as a backend for other applications, we do not want to store both the
username and the password of the user locally on the device for the entire
runtime of the application. HTTP basic authentication requires sending the
plaintext username and password with every request, which would necessitate
storing those credentials. To avoid the security implications of that storage,
we have chosen to implement method 3, where we use the HTTP standard to its full
potential, but define our own authorization scheme, which will not require
storing or sending the username and password for every request.\cite{http_basic}
A problem with method 3 is that the client accessing the API has to be able to
set the authorization scheme to something custom. While this requirement makes
is harder to explore the API in a web browser, it is our judgement that most
application frameworks and tools will let you set the authorization header
manually.

The standard for defining a custom HTTP authorization scheme is to accept some
data in the HTTP \code{Authorization} header. The header value has to, by
convention, start with the name of your scheme, followed by a whitespace and
then data of the schemes choice\cite{http_basic}. Since we want a token based
authentication, to avoid the client having to store the password, we decide that
our authorization header will look like the following: \code{Sleepy
token=<tkn>}. Where token is the unique token for the current session of
a specific user, it is notable that a user can have multiple sessions open at
once.

\subsection{Filter}
\label{subsec:auth_filter}

JAX-RS, and RESTEasy, has support for roles and authorization via the
\code{RolesAllowed} annotation, which is put on the service methods that need
protection. The JAX-RS framework then asks the security context if the current
request is allowed. We set the security to a custom subclass that implements our
permission system, using a \code{ContainerRequestFilter}. The filter is run
before much of the RESTEasy processing, where it extracts the header, looks for
the token in the database, and sets the security context. The JAX-RS framework
asks the \code{SecutityContext} if the authenticated user is authorized to take
a given role. If it is not, the framework will immediately conclude the request
with an appropriate HTTP response code. If the user is allowed however, the
request is continued as normal. The JAX-RS authentication model we have chosen
is entirely transparent to the REST endpoints once they have declared the roles
allowed. It also makes it possible to add another authentication method, such as
another one listed above, with no development overhead, thus allowing for great
functional scalability.

We have also decided to make the authentication filter inject the authenticated
user into the container. This facilitate interaction with the logged in user in
methods further in the chain. A diagram over the authentication filter can be
seen in \cref{fig:auth_auth}.

\subimport{}{auth_seq_auth.tex}

\subsection{Endpoint}
The filter described in \cref{subsec:auth_filter} looks for an authentication
token in the database. To create an authentication token we need a REST endpoint
to insert the token into the database, and associate it with a user.

We want to expose three basic authentication operations to the users: login, logout,
and a listing of all the currently active authentication tokens. As logout is
non--essential and trivial, we have decided not to implement it. That leaves us
with login and listing. We decided to map the operations to REST endpoints as
can be seen in \cref{tab:rest_auth}. The \code{POST} request to \code{auth/} is
what generates a new token, inserts it into the database, and then inserts it to
the client. We have implemented the expired property on the \code{AuthToken}, to
allow for easier implementation in the future, without breaking database, and
client, compatibility. The \code{GET} request lists the other tokens for the
user currently logged in. In the table the unimplemented \code{DELETE} method is
also shown, this endpoint would immediatly expire the token with the id
\code{<aid>}.

\begin{table}[htpb]
    \centering
    \begin{tabu} to \textwidth {lX[c]l}
        Endpoint                         & HTTP Method & Description                                     \\
        \midrule
        \rendpoint{auth/}                & \code{GET}         & List the tokens                          \\
        \rendpoint{auth/}                & \code{POST}        & Create new token                         \\
        \rendpoint{auth/<aid>}           & \code{DELETE}      & Expire a token (not implemented)         \\
    \end{tabu}
    \caption{Authentication endpoints.}
    \label{tab:rest_auth}
\end{table}
