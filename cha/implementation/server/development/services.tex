\subsection{Services Overview}\label{sec:services_overview}
This section provides an introduction and overview of the services we have created for the server.

A service delivers one or more resources which in turn contains end--points.
These end--points are the interface to the server which is used by the developer of the producer and/or consumer clients.

Each service is a Spring Bean, and is injected with the correct context when required. \fxnote{Er ikke 100\% på termonologien når det kommer til spring mumbo jumbo.}

\subsubsection*{Overview of API End--points}
We have 11 end--points split over four services and two sub--services.
A sub--service is a service which is a constituent of another service, i.e. it is served by it.
An overview of these are shown in \cref{table:endpointservice}.
The four services are \code{AuthService}, \code{UserService}, \code{VehicleService}, and \code{RouteService}, 
which all serve end--points relative to the root of the API.
There are also two sub--services, they are \code{WaypointService}, and \code{VehicleDataService}.
They are served as sub--services of the \code{RouteService}, since they are reliant on a \code{Route}.
Each of the services and sub--services are explained later in further detail.
The ``More Info''--column of the table lists the page of each.

We decided not to implement any deletions in the REST API.
Implementing this would have taken development time away from what we have considered more important features. \fxnote{Tilføj dette til future work.}
\subimport{dankmemes}{services_overview_table.tex}

\subsubsection*{Implementing Services}
Each HTTP method for each endpoint needs to be implemented as a method in Java.
For this we use JAX--RS, as introduced in \cref{subsec:tech_communication}, which provides annotations, such that the application server can handle incoming requests.
We use the annotations in \cref{table:serviceannotations} most often in our services.

\subimport{dankmemes}{services_annotation_table.tex}

\bigskip
To illustrate this we provide an example in \cref{lst:userServiceGetById}.
In this example we fetch a user from the database and return it as JSON, if certain conditions are met.
First on line 1--4 we have annotations, as explained in \cref{table:serviceannotations}.
Then on line 5, we use the \code{@PathParam} to access the variable defined in the \code{@Path}--annotation on line 2.
On line 6, we use the \code{UserDao}, explained in \cref{subsec:persistingtodb}, to fetch the user.
The method is named \code{byId\_ForDisplay} since it initializes the fields which are lazily loaded, such that they can be sent to the user.
On line 8, we check if a user with the given id is found, if not, then we return an error.
Finally on line 11 we return the user object we fetched from the database.
Then, as explained in \fxnote{Serialization}, it will be serialized into JSON.
\begin{listing}
    \begin{java2}
@GET
@Path("/{uid}")
@Produces("application/json")
@RolesAllowed({PermissionType.Constants.USER})
public User getUserById(@PathParam("uid") Long id) {
    User user = userDao.byId_ForDisplay(id);

    if (user == null)
        throw new NotFoundException("User not found.");

    return user;
}
    \end{java2}
    \caption{A sample method from the \code{UserService}, which fetches a single user by id.}\label{lst:userServiceGetById}
\end{listing}

After any creation, we return the created object. 
Otherwise the client would not have an id to refer to the object.
We do the same when a client updates an object, 
since another client might have changed some other fields since the object was last fetched. 

\bigskip
\paragraph*{Error messages}
In order to ease debugging for the users of our API, we want to provide as precise and detailed error messages as possible,
we provide these messages as a JSON object.
We also return an appropriate HTTP status code, the most common ones we use for reporting errors are:
\begin{multicols}{3}
\begin{itemize}
    \item 400 Bad Request,
    \item 401 Unauthorized,
    \item 403 Forbidden,
    \item 404 Not Found,
    \item 409 Conflict.
\end{itemize}
\end{multicols}

For example, if insufficient arguments were given, a ``400 Bad Request'' would be the response,
but additionally the missing argument(s) are listed.
For ``409 Conflict'', it is also important to provide information about what conflicted,
for example when trying to create two vehicles with the same VIN.

\paragraph*{Builders}
When a client makes a POST or PUT request for a given object, 
the server also requires some additional information about the relation the object has to other classes, 
e.g. a route would require driver and vehicle information.

Providing entire objects of each relation would be impractical for the client as they might be outdated and it would increase the size of each request.
Ideally the client would send the minimal amount of information, i.e. just the ids of the objects.

To facilitate this we create builders, following the Builder pattern from ``Effective Java''\cite{Bloch:2008:EJ:1377533}. % http://www.javaworld.com/article/2074938/core-java/too-many-parameters-in-java-methods-part-3-builder-pattern.html

\bigskip
A builder is a class which is similar to the class it builds, has the same properties, with the following changes:
\begin{itemize}
    \item all references to existing items in the database are replaced with ids,
    \item all properties should have public getters, and setters,
    \item the id field is excluded.
\end{itemize}

Additionally all builders must have a build method, which build the object of the actual class, and returns it.
If any fields are ids then building the objects will require accessing the database,
therefore the build methods do in certain cases take \acp{DAO} as parameters.
The \acp{DAO} are used to fetch the full elements from the database, 
such that the appropriate relations can be set.

We also use builders to modify existing objects,
in that case there is a method which takes the object we want to modify,
then on that object it will replace all the fields which has been set in the builder.

\subsubsection*{Documenting Services}
In order to document the services we use Enunciate\footnote{\url{https://github.com/stoicflame/enunciate}}.
Enunciate is a tool, which among other features, can automaticly generate documentation for a REST API written in Java like ours.
It does so by inspecting the code, the annotations, and our doc comments etc.

Doc comments\footnote{\url{http://www.oracle.com/technetwork/articles/java/index-137868.html}} is a specific style of comments in Java, they are sometimes also refered to as JavaDocs, which is a tool developed by Oracle to generate documentation by analyzing doc comments.
Doc comments are placed immediately above the method, field or class they are related to.
The doc comments for the method listed in \cref{lst:userServiceGetById}, is shown in \cref{lst:userServiceGetById_comments}.
Doc comments explain what the method does and what it will return.
This includes the parameters, the return value, and for any exceptions which can be thrown.

\begin{listing}
    \begin{java2}
/**
 * Get a user by their id.
 *
 * @param id The id of the {@link User user} to retrieve
 * @return The retrieved {@link User user}
 *
 * @HTTP 404 User not found
 */
    \end{java2}
    \caption{The doc comments for the method listed in \cref{lst:userServiceGetById}.}\label{lst:userServiceGetById_comments}
\end{listing}

The documentation generated by Enunciate shows all the end--points, as well as how they are used,
with JSON examples of the data sent and retrieved from them.
We have included the entire Enunciate documentation on the ZIP file.\fxnote{Inkluder hele enunciate i bilag cd zip fil}

