% vim: spell;tw=80
\section{User endpoints}

In \fxnote{sec:requirements} we identified the need of access management in the
both the consumer and provider client. To facilitate the features in the
front--end, the server will keep track of users, permissions, and passwords.
Users and their permissions are also what is used in the authentication step
developed in \fxnote{authentication}. The prototype consumer, developed in
\fxnote{front--end chapter}, also required icons for users.


From these requirements we derived the endpoints shown in \cref{tab:rest_user}.
All the user endpoints require at least \code{User} permission, except the
endpoint to create and modify a user, which requires \code{SuperUser}
permission. The create and modify endpoints both use the builders discussed in
\fxnote{ref til Builder}. The \code{list all} endpoint gets a list of users,
but is not guaranteed to include all details about them, in the current version
they do, but more details could be shown only in the detailed \code{Get user}
endpoint, to conserve bandwidth and database joins. To enable the storage of
user icons we use the \code{FileDao} from \fxnote{ref til faildao database},
which allows for easy storage of binary files in the database. We are currently
only allowing files in the \code{image/png} \ac{MIME} format, but any binary
file can be uploaded.  If the system was to be deployed in production, a limit
on the size or file type would have to be placed in the user icons. For the
endpoint to upload a user icon, we chose to use the HTTP method \code{PUT},
since the request is idempotent, in other words, doing the operation, uploading
the same image, twice does not change the state of the object. We have allowed
the user icon be downloaded as \code{BASE64}, to help consumer developers
wanting to download the image from JavaScript, while setting the headers, and
displaying the resulting image in a browser.

\begin{table}[htpb]
    \centering
    \begin{tabu} to \textwidth {lX[c]l}
        Endpoint                         & HTTP Method        & Description                              \\
        \midrule
        \rendpoint{user/}                & \code{GET}         & List the users                           \\
        \rendpoint{user/}                & \code{POST}        & Create new user                          \\
        \rendpoint{user/<uid>}           & \code{GET}         & Get user                                 \\
        \rendpoint{user/<uid>}           & \code{PUT}         & Modify user                              \\
        \rendpoint{user/<uid>/icon}      & \code{GET}         & Get user icon                            \\
        \rendpoint{user/<uid>/icon}      & \code{PUT}         & Set user icon                            \\
    \end{tabu}
    \caption{User endpoints.}
    \label{tab:rest_user}
\end{table}
