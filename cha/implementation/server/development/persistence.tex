% vim: tw=80:spell:spelllang=en
\subsubsection{Persistence}
The persistence layer is what persists the object between requests and
sessions. In effect it's where all the state is stored and retrieved. To store
the needed data, the layer makes use of \code{Hibernate-OGM},
\code{Hibernate-Search}, and \code{Spring}. We have chosen to use the JPA
instead of using hibernates own interfaces, since that gives us the opportunity
to switch out Hibernate in the future.

A large part of the configuration is done in the persistence layer, since this
is where we define our connections to \code{Mongodb} and configure the hibernate
library to use it. Our configuration of hibernate is done with spring, via the
\code{persistence-config.xml} file. Doing the configuration via spring achieves
two specific goals. It makes configuration easily modified, since it's all
specified in a single XML file, rather than being spread throughout the
codebase. It also helps integrate hibernate with spring, allowing spring to
manage the \code{EntityManager}, and inject it where necessary. Spring will also
manage things like transactions this way.

For persistence i want to know
\begin{itemize}
    \item What is the purpose of this layer?
    \item What does this layer contain? (take folders and explain contents of folders using 1 file from each)
    \item What issues does this layer handle?
    \item How is this layer accessed (communication between layers)?
    \item How are the technologies from the techstack used?
    \item How is the architecture and design pattern present?
    \item Hibernate magix, wut is dis
\end{itemize}
