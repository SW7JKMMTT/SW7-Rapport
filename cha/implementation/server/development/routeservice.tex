\section{The Route Services}
In this section we describe the functionality, and choices made when developing the \code{RouteService}, 
and its two sub--services \code{WaypointService}, and \code{VehicleDataService}.

In \cref{table:endpointrouteservice} we provide an overview of the endpoints in the \code{RouteService}, \code{WaypointService}, and \code{VehicleDataService}. 
In the last column of the table we list the name of the method which handles the request in the corresponding service. 

\subimport{dankmemes}{route_services_table.tex}

\subsection{RouteService}
The most interesting and complex method in the \code{RouteService} is the \code{getAllRoutes} method.
It is more complex since there are more cases which much be handled. 
If no query parameters are set, then it will return all the existing routes.
However it can take five optional parameters: \code{routeState}, \code{driverId}, \code{latitude}, \code{longitude}, and \code{radius}.

\code{routeState}, and \code{driverId} are filters, meaning that if one of them is set then it will be filtered by it, 
they can also be used in conjunction.
The three remaining parameters are used for spatial queries, if one of them is set, then all of them must be set for the query to make sense.
We have previously explained how spatial queries are created and executed in the \ac{DAO}, in \fxnote{Kan ikke ref til den da den ikke har et nummer, men det bliver fixet senere får label: subsec:spatial}.

The spatial queries can also be used in conjunction either one or both of \code{routeState}, and \code{driverId}. 
In total there are eight methods in the \ac{DAO}, which can be called from this method depending on how many of the optional query parameters are set.
A simpler method for filtering would be to fetch all elements from the database, then apply filters. 
However, that method would cause unnecessary load on both the database and our application, which is unwanted. 
Especially as the number of elements in the database increase, thus it would worsen the load scalability of the system. 
We chose to use the \ac{DAO} for this filtering since we want to utilise what the database is good at, such as filtering large amounts of data.

\bigskip
The remaining methods in the \code{RouteService} are fairly straight forward.
% Jeg relaterer dem her til deres metoder, måske man skulle relatere dem til endpoints i stedet? Opinions? Tabellen relaterer jo også de 2.
\code{createRoute} takes a \code{RouteBuilder} which contains a \code{vehicleid}, \code{driverid}, and \code{routeState}.
If all required fields are set, i.e. \code{vehicleid}, and \code{driverid}, then it will continue otherwise it will give an error.
Then \code{createRoute} fetches the actual \code{Vehicle} and \code{Driver} with their appropriate \acp{DAO}, and uses the \code{RouteBuilder} to build the \code{Route}, 
which it then adds to the database via the \code{RouteDao}, and then the method returns the \code{Route} created to the user. 

\code{modifyRoute} is very similar to \code{createRoute}, with the exception that the method fetches an existing object, 
then the method uses the \code{RouteBuilder} to modify the fetched object, instead of creating a new one.
This also means that not all fields must be set on the \code{RouteBuilder} as they were in the context of creating a new one.

\code{getRouteById} fetches, and returns a specific route from the \ac{DAO} by its id.

\subsubsection*{Serving Sub--Services}
Lastly in the \code{RouteService} are two methods each of which are responsible for serving a sub--service.
They both work in the same way, first they fetch the \code{Route} via its id, then the methods sets up their appropriate sub--service, and then serve it.
The route upon which a sub--service is working, is stored as a local variable in the sub--service, thus it is stateful. 
Therefore the sub--services must be created before each request.
This is done by making the scope of the sub--service last for the duration of the request, i.e. request scoped.
In contrast to the default which is application scoped, meaning that the services are created once and lasts for the duration of the application.
Setting a service or sub--service to be request scope is done by setting the \code{@Scope} annotation and giving it the value ``request''.

For the sub--services we do not initialize the collections of waypoints or vehicledatapoints on them.
If we eagerly fetch the collections on a \code{Route}, then the time it took to create a sub--service would be linear w.r.t. the size of the collections.
Thus the performance would degrade as more \code{Waypoint}s, and/or \code{VehicleData} are added to \code{Route}s.

\subsubsection*{WaypointService}
The \code{WaypointService} is used to get and post waypoints to a route.
It contains two methods, \code{getWaypoints}, and \code{addWaypoint}.

\code{addWaypoint} takes a \code{WaypointBuilder} if the appropriate fields are set, then it will construct the waypoint, and add it to the route.
If not then it will return an error.

\code{getWaypoints} takes two optional query parameters: \code{maxCount}, and \code{after}.
\code{maxCount} is the maximum number of waypoints to return.
\code{after} is a timestamp which is used to only fetch waypoints newer than the timestamp.
These two can be used in conjunction, meaning that there are four combinations in total.
This is powerful for the consumer clients in many cases.
For example to use for a map, only new information is required, thus this can avoid a lot of redundant data transfer.

\subsubsection*{VehicleDataService}
The \code{VehicleDataServer} is used to get and create vehicle--data to a route.
It has two methods for this: \code{getVehicleData} and \code{addVehicleData}.

\code{getVehicleData} returns the vehicle--data stored on the route.
This is fairly straight forward. 

\code{addVehicleData} is more complex.
It uses a \code{VehicleDataBuilder}, which consists of a timestamp and one or more \code{VehicleDataPoint}s.
We have previously documented how the relationship between \code{VehicleData} and \code{VehicleDataPoint}s in \fixme{Kan ikke ref før refactor: par:vehicledata}.
We do not use a builder to construct the \code{VehicleDataPoint}s, but instead we use annotations which are part of Jackson, which we described in \fixme{Tech stack efter update???}.
This is done since it is easier to add more different \code{VehicleDataPoints}, since they would only require changing the model, which we described in \fixme{Kan ikke ref lige nu men det bliver par:vehicledata}.
