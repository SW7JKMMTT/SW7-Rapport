% vim: tw=80:spell:spelllang=en
\subsubsection{Core}
In the core layer we implement all the business logic and data, which is
independent from the presentation of that data. As stated in~\ref{subsec:internal_architecture},
we want to keep the models in this layers closely related to our business
constraints. As such we want to keep members private, and only provide access to
the properties that make sense from a business model perspective. To facilitate
separation, we don't allow direct access to list, opting instead to wrap them in
\code{ReadOnlyCollection}. Mutation of collections have to be performed by
accessing methods on the object that owns the collection.

By making these choices we can sufficiently control what can and can't be done
to our model by varying the accessibility of the properties and methods. We use
the accessibility modifier to help us maintain the invariants in our model, such
as a route always being driven by a vehicle. Without the constraints we impose
via our accessors and implemented constructors the later modules could create
a route that isn't being driven by any vehicle. Since that doesn't make any
sense from a modeling perspective, we try to disallow that at compile time.

Since classes in the core module are mostly intended to hold data about the
model, and should be independent from the presentation and storage of the
object, they are mostly implemented as POJOs.

\fxnote{TODO:\@ Make some diagrams!}

For the core i want to know
\begin{itemize}
    \item What is the purpose of this layer?
    \item What does this layer contain? (take folders and explain contents of folders using 1 file from each)
    \item What issues does this layer handle?
    \item How is this layer accessed (communication between layers)?
    \item How are the technologies from the techstack used?
    \item How is the architecture and design pattern present?
\end{itemize}
