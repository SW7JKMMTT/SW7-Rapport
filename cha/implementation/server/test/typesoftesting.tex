\section{Types of Testing}
Ian Sommerville provides three stages of testing a system\cite[p.~231]{software_engineering}:
development testing, release testing, and user testing.
Development testing is tests made by the developers of the system during development
and includes unit testing, regression testing, component testing,
and system testing\cite[p.~232]{software_engineering}.

The second stage is release testing,
where a particular release of a system is prepared and tested to prove it is ready for use outside of the development environment\cite[p.~245]{software_engineering}.
Release testing includes requirements--based testing, scenario testing and performance testing.
The final test stage is user testing where users or customers provide feedback to the system.\cite[p.~249]{software_engineering}
User testing includes alpha testing, beta testing and acceptance testing.

User testing is deemed not to be relevant in testing the server,
since the server provides data for a front--end,
and an end user would only interact through a client with the server.
If we were to make a front--end, which is not only a proof of concept, user testing would be relevant.

We will perform development testing to test the server.
This choice is taken since we find elements from development testing relevant.
The methods we will use from development testing:
unit testing, regression testing, and component testing.

The form of release testing which Ian Sommerville describes do not fit very well with our development method.
This way of release testing is for major releases, where we instead create small and continuous releases.\fxnote{Vær sikker på at dette bliver nævnt tidligere.}
The way we will do release testing is by doing code review on code changes and make sure all development tests pass.
Nevertheless we will still do performance testing, not in the context of releasing, but to benchmark the server.

\bigskip

Another type of test we perform is smoke testing.
Smoke testing is an ad-hoc testing method where basic functionality is tested and happens naturally during development\cite{smoke_testing}.
An example of smoke testing in a software system is to make a compilation and see if it succeeds, since we use a statically typed language,
we can use the type system to check if all types are compatible.

To sum up, we will do unit testing, regression testing, and component testing which is a part of development testing.
Furthermore we will do our own version of release testing, do performance benchmarking,
and we will use smoke testing to quick test basic functionality.
