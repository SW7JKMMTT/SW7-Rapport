\section{Code Coverage}\label{sec:code_coverage}
It is difficult to determine when a piece of software is adequately tested.

A way of measuring how much the code in a software project is tested is called code coverage.
There exist different metrics to measure code coverage, e.g. statement coverage and condition coverage,
and which one to use should be chosen according to the specific use case.
Performing a code coverage analysis can pinpoint untested areas of a software project.
It also calculates a quantitative measure of the code coverage,
which can be used to estimate if more tests are needed or not,
but it do not say what must be tested.\cite{code_coverage}

The downside of using code coverage as a goal for testing,
is that the code coverage measure of a software project does not say anything about the quality of the tests,
only that the code is being tested.
Therefore relying too heavily on having a high code coverage can give a false impression of the quality of the software.\cite{code_coverage_neg}

In a larger system having a code coverage of 100 \% is generally impractical and often it is not cost effective compared to the benefit gained by testing.
It is also estimated that a code coverage at 100 \% only expose about half the faults in a system.
Instead of testing everything the focus should be at finding bugs.
The point at which to stop testing is when the tests become contrived.\cite{code_coverage_not_100}

\bigskip

In regards to testing the server, we will not use code coverage as a goal for testing.
This choice is taken since code coverage do not give any qualitative information of the testing.
Instead code coverage will be used to find areas with low code coverage or no tests.
These areas will then be investigated to see if they need more tests or if the testing is adequate.

From the previously mentioned reason for testing,
it is stated that the components, which makes the software fulfil its requirements, should be tested.
Therefore when testing the server, the functionality,
which directly fulfil the requirements stated in \cref{cha:requirements_elicitation}, will be tested.

We do not deem it sufficient to only test the functionality which directly fulfil the requirements of the server.
Therefore selected parts of the server, which indirectly fulfil the requirements,
will also be tested to make sure they behave correctly.
The selected parts will be chosen from their importance in the functionality of the server,
for instance testing database connections,
which is not directly a part of the requirements from \cref{cha:requirements_elicitation}.

\fxnote{Dette skal revideres når de endelige krav kommer, f.eks. ryger det med databasen nok.
Alternativt kan det med indirekte krav slettes hvis det ``indirekte krav'' ikke anses som passende og så testen tilpasses så der kun testes for kravene.}

Thorough testing of the server will also help avoid regression during development.
Regression bugs can be hard to debug, since the system initially worked correctly and suddenly it does not.
With a combination of unit tests and integration tests together with the automatic running of tests,
the risk of not catching regression bugs is minimised\cite{regression}.

To avoid regression in the code, the execution of tests should be done automatically.
The reason for this is that a software projects can have as many good tests to test the code,
but if the developers forgets to run the tests during development,
the tests will be useless.

\bigskip
To summarize, we will test that the server satisfies the requirements,
and functions as intended.
Furthermore we will test other functionality which indirectly fulfil the requirements,
to avoid bugs and regression during development.
Finally the tests will be run automatically to make sure that regression bugs and other bugs are caught.
