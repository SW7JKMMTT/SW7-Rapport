\section{Internal Tests}
It is not possible to achieve an adequate level of testing of the server, using only one type of test.
This is caused by the fact that the server is made up of different components,
as mentioned in \cref{sec:internal_architecture},
which depends on and uses each other.
Furthermore testing the different aspects of the requirements, for instance ``Support load scalability.''
and ``Normalise incoming data and offer the data in a platform neutral format.'' puts completely different
requirements on the tests and testing methodology.
Therefore several testing methods is used to test the requirements to the server
and the important functionality to achieve an adequate level of testing.
The different testing methods we will use to test the server with is unit, integration, load, and system testing.

\subsection{Unit and integration testing}
For unit and integration testing, JUnit\footnote{\url{http://junit.org/junit4/}} with Hamcrest\footnote{\url{http://hamcrest.org/}} are used.
JUnit is a testing framework and Hamcrest is a matching framework, and therefore they have different types of assert methods.
Where JUnit have assert methods like \code{assertEquals}, Hamcrest have \code{assertThat}.
Since JUnit 4, Hamcrest's \code{assertThat} have been a part of JUnit and can be used without including Hamcrest separately\cite{hamcrest_vs_junit}.

The advantage of using \code{assertThat} over JUnit's assert methods is firstly the readability.
An example of the improved readability can be seen in \cref{lst:assert_readability},
where the \code{assertThat} statement at line 1 can be read as plain English,
but the \code{assertFalse} statement at line 3 can be harder to interpret for someone unknown to the code\cite{hamcrest_vs_junit}.

\begin{listing}
    \begin{java2}
        assertThat(actual, is(not(equalTo(expected))));

        assertFalse(expected.equals(actual));
    \end{java2}
    \caption{An example of the difference in readability taken from \cite{hamcrest_vs_junit}.}
    \label{lst:assert_readability}
\end{listing}

Furthermore the \code{assertThat} statement gives better failure messages\cite{hamcrest_vs_junit}.
An example of that can be seen in \cref{lst:assert_diff},
where a failed \code{assertTrue} returns the failure message at line 1 to 8 (only relevant information is showed),
which do not contain the values, which failed the test.
A failed \code{assertThat} returns the failure message at line 10 to 14,
but this message contains the values which failed the test,
which is helpful in debugging the reason the test failed.
It also makes it easier for people relatively unknown to the tests to debug a failed test.

\begin{listing}
    \begin{java2}
    junit.framework.AssertionFailedError
        at junit.framework.Assert.fail(Assert.java:55)
        at junit.framework.Assert.assertTrue(Assert.java:22)
        at junit.framework.Assert.assertTrue(Assert.java:31)
        at junit.framework.TestCase.assertTrue(TestCase.java:201)
        at rocks.stalin.sw708e16.server.persistence.TestSpatialRouteDao
            .testWithinRadius_WrappingAroundGlobeLongitude(TestSpatialRouteDao.java:225)
        ...

    java.lang.AssertionError:
    Expected: not a collection containing <rocks.stalin.sw708e16.server.core.spatial.Route@544e3900>
    but: was <[rocks.stalin.sw708e16.server.core.spatial.Route@544e3900]>
    Expected :not a collection containing <rocks.stalin.sw708e16.server.core.spatial.Route@544e3900>
    Actual   :<[rocks.stalin.sw708e16.server.core.spatial.Route@544e3900]>
    \end{java2}
    \caption{An example of the difference in failure messages.}
    \label{lst:assert_diff}
\end{listing}

To test the individual components of the server,
a combination of unit tests with JUnit and Hamcrest is used.
An individual component is a component which do not depend or interact with other components.

An example of a unit test from the server can be seen in \cref{lst:unit_test_example}.
The test checks if the user icons can be set on a user and retrieved.
First at line 4 a user is initialised without a user icon.
At line 7 a user icon is assigned to the created user,
and at line 10 it is tested that the user actually has the icon.
At line 1, the annotation \code{@Test} is JUnits way of defining a method as a test.

\begin{listing}
    \begin{java2}
        @Test
        public void testHasIcon_WithIcon_IsTrue() throws Exception {
            // Arrange
            User jeff = new User("Jeff", "passw0rd", "Jeff", "Lam");

            // Act
            jeff.setIcon(new UserIcon());

            // Assert
            assertThat(jeff.getHasIcon(), is(true));
        }
    \end{java2}
    \caption{An example of a unit test from \code{UserTest.java} which is a part of Core.}
    \label{lst:unit_test_example}
\end{listing}

As can be seen in \cref{lst:aaa_example}, we use the \ac{AAA} pattern when doing unit tests.
The \ac{AAA} pattern splits the tests up into three parts.
First is the Arrange part, where objects are initialised and variables declared and assigned.
Second is the Act part, where the invocation of the method which is tested is happening.
Lastly is the Assert part, where it is verified that the method which is tested is behaving as expected.\cite{aaa_pattern}
The test seen in \cref{lst:aaa_example} tests if a created user do not have a \code{authToken} as default.

\begin{listing}
    \begin{java2}
        @Test
        public void testListTokens_NoTokens() throws Exception {
            //Arrange
            User jeff = new GivenUser().withName("Jeff", "Jeffsen")
                .withUsername("Jeff").withPassword("password").in(userDao);

            // Act
            Collection<AuthToken> authTokens = authService.listTokens(jeff);

            // Assert
            Assert.assertTrue(authTokens.isEmpty());
        }
    \end{java2}
    \caption{An example of the use of \ac{AAA} pattern from \code{TestAuthenticationService.java} which is a part of Service.}
    \label{lst:aaa_example}
\end{listing}

\bigskip

When the components to be tested require more than them self to be tested,
for instance a database connection,
then integration testing is used.
The way the server is integration tested is the same way as unit tested,
except that the test depends on other components.
An example of this can be seen on \cref{lst:integration_test_example},
where at line 3 the \code{waypointDao} is used.
The \code{waypointDao} is used to communicate with the database,
and therefore this is an integration test instead of a unit test.

\begin{listing}
    \begin{java2}
        @Test
        public void testWithinRadius_WithNone_FindNone() throws Exception {
            List<Waypoint> ret = waypointDao.withinRadius(0, 0, 100);

            Assert.assertThat(ret, hasSize(0));
        }
    \end{java2}
    \caption{Integration test example from \code{WaypointDao.java} in Persistence.}
    \label{lst:integration_test_example}
\end{listing}

\subsubsection*{The \code{@Before} annotation}
To make sure the state of the program is the same before each test and all resources needed are available,
the JUnit annotation \code{@Before} is used.
\code{@Before} is used to run methods before each test is executed,
for instance to set up needed objects\cite{before_doc}.
An example of the use of \code{@Before} in the server is to clear the test database,
to make sure the state of the database is the same before each test.
This is done by having all the test classes, which use the database, inherit the abstract class \code{DatabaseTest.java},
where the method \code{clearDatabase}, seen on \cref{lst:cleardatabase}, is defined.
The database is cleared before each test to make sure that each test has no assumptions about other tests,
since they are unordered and each should stand for itself.

\begin{listing}
    \begin{java2}
        @Before
        public void clearDatabase() {
            MongoClient client = new MongoClient(informationProvider.getUrl(), informationProvider.getPort());
            client.getDatabase("test").drop();
        }
    \end{java2}
    \caption{\code{clearDatabase} method from \code{DatabaseTest.java}.}
    \label{lst:cleardatabase}
\end{listing}

Another use of \code{@Before} in the server is to set up a waypoint service before there is made test on it.
The method \code{SetupWaypointService} from \code{TestWaypointService.java},
which can be seen in \cref{lst:setupwaypointservice},
creates a user, driver, vehicle, a route, and then sets the route.
The \code{setRoute} method at line 13 configures the \code{waypointService} to use the route \code{route}.
\code{TestWaypointService.java} also inherits \code{DatabaseTest.java},
so the way \code{@Before} works with multiple \code{@Before} statements in different classes
is that the \code{@Before} methods of the superclass will be executed before the \code{@Before} methods of the current class\cite{before_doc}.
This is convenient with the way we have constructed our tests since the newly inserted data will not be removed.

\bigskip

When test data is needed by the tests, test data is created and inserted in the database by using the \code{Given} classes.
An example of this can be seen in \cref{lst:setupwaypointservice}.
Using the \code{Given} classes gives a easy and transparent way of creating test data for the tests, which is also their only purpose.
By creating the \code{Given} classes as fluent interfaces\footnote{\url{http://martinfowler.com/bliki/FluentInterface.html}}, the method calls can be chained as shown in \cref{lst:setupwaypointservice} line 3 to 12,
thus the code can be read as plain English.

\begin{listing}
    \begin{java2}
        @Before
        public void SetupWaypointService() {
            User user = new GivenUser().withName("Jeff", "Jeffsen")
                .withUsername("Anders").withPassword("hunter2").in(userDao);
            Driver driver = new GivenDriver().withUser(user).in(driverDao);
            Vehicle vehicle = new GivenVehicle()
                .withMake("Ford")
                .withModel("Mondeo")
                .withVintage(1999)
                .withVin(new Vin("ABC123"))
                .in(vehicleDao);
            Route route = new GivenRoute().withDriver(driver).withVehicle(vehicle).in(routeDao);
            waypointService.setRoute(route);
        }
    \end{java2}
    \caption{\code{SetupWaypointService} method from \code{TestWaypointService.java} in Services.}
    \label{lst:setupwaypointservice}
\end{listing}

\subsection{Testing status}
To give an overview of the testing status of the software, we will look at the tests in Persistence and Services.
We will not look at the testing of Core, because Core do not contain any business logic.

In \cref{tab:test_cov} the method coverage of the methods in Persistence and Services can be seen.
Method coverage measures that a method have been tested, e.g. a class with two methods, where one is tested would have a 50 \% method coverage.
Furthermore the number of tests which tests the Persistence and Service can be seen.
Since the relation between the classes in Persistence and Services and their test classes are not one--to--one, i.e. a test class do not only test a single class,
the number of tests for each class is listed as N/A.


\subsubsection{Persistence}
Persistence contains the \acp{DAO}, as mentioned in \cref{sec:internal_architecture}.
Each \ac{DAO} has a method coverage of 100 \%, as seen in \cref{tab:test_cov}.
Not everything have a code coverage of 100 \% since not all methods are relevant to test, 
but the method coverage is at 94.7 \%, as seen in \cref{tab:test_cov}.
This only states a quantitative measure of the extend of the testing. 

To make sure the tests are adequate, i.e. they tests a broad spectrum of use cases and verify the correctness,
the \acp{DAO} are tested under different conditions.
Among these conditions are having different data in the database and testing with different input parameters.

\subsubsection{Services}
In the Service component, the business logic is tested.
More precisely is the public \ac{API} of the server tested, without testing the used frameworks.
We do not test the frameworks, since we expect and trust that they work as expected.

Not every method in Service is tested, since as with Persistence, since not all methods are relevant to test.
Still Service have a method coverage on 95 \%, as seen in \cref{tab:test_cov}.

As with Persistence, we test the Service component under different use cases, for instance with different input parameters,
with the database containing different data, and also tests with malicious input, to see if the system handles it.

It should also be noted that the \acp{DAO} are also tested by other tests than the only made directly for them, 
as they are integration tests.

\begin{table}
    \center
    \begin{tabular}{l|l|l}
        \textbf{Component}      & \textbf{Method coverage \%} & \textbf{Number of tests} \\
        \toprule
        Persistence             & 94.7  & 54 \\
        \toprule
        AuthTokenDaoImpl        & 75    & 5 \\
        BaseDaoImpl             & 80    & All\\
        DriverDaoImpl           & 100   & 4\\
        PermissionDaoImpl       & 100   & 0\\
        RouteDaoImpl            & 72.7  & 17\\
        UserDaoImpl             & 85.7  & 1\\
        VehicleDaoImpl          & 100   & 7\\
        MiscDataDaoImpl         & 100   & 1\\
        WaypointDaoImpl         & 100   & 18\\
        \midrule
        Services                & 95    & 100 \\
        \toprule
        AuthenticationService   & 66.7  & 11 \\
        RouteService            & 75    & 35\\
        UserService             & 71.4  & 20\\
        MiscDataService         & 100   & 5\\
        VehicleService          & 100   & 22\\
        WaypointService         & 100   & 9\\
        \bottomrule
    \end{tabular}
    \caption{Test coverage of Persistence and Services.}\label{tab:test_cov}
\end{table}
