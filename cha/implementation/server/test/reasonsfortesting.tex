\section{Reasons for Testing}
The intention of testing software is to demonstrate that it does what it is intended to do.
According to Ian Sommerville in his book \textit{Software Engineering}\cite[p.~227]{software_engineering} testing is supposed to do two things:

\begin{description}
    \item[Validation testing]\cite[p.~227]{software_engineering} \hfill\\
    To test if the software meets its requirements validation testing is used.
    This means that there should, for a custom system, be a test for each requirement,
    demonstrating the ability of the software to fulfil the requirements.
    The test cases to test the requirements is supposed to reflect the intended use of the software,
    i.e. the tests should imitate regular use from a regular user.

    \item[Defect testing]\cite[p.~227]{software_engineering} \hfill\\
    To find input which makes the software behave incorrectly defect testing is used.
    The incorrect behaviour is the result of software defects,
    which can result in system crash, data corruption or other faults.
    Testing for incorrect behaviour also prevent regression during development,
    where new bugs could be introduced.
    Regression is the case where a piece of working software stops working after a new feature is added\cite{regression}.
    The test cases to expose the defects are not required to imitate regular use as validation testing.
    Therefore these test cases can be obscure, such that edge cases can be covered.
\end{description}

Since we are interested in the server being able to fulfil its requirements,
and we want to find and eliminate incorrect behaviour, we determine that we have to test the server.
Furthermore testing is a part of the requirements of the server, as mentioned in \cref{cha:requirements_elicitation}\fxnote{Add testing som requirement}.
Since our core area, as mentioned in \cref{sec:problem_statement} is the server, we will not test the front--end.
The next sections will describe the testing methods we intend to use.

