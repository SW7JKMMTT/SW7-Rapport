\subsection{System and Load Testing}
\begin{itemize}
    \item What is this shit even? \checkmark
    \item What is our strategy, what outcome do we want?
    \item Show me some numbers and graphs, beautiful graphics
    \item Interpret these beautiful graphs and numbers for me, i dont know math.
    \item How do the results reflect on our requirements?
\end{itemize}

To test our server application as a whole, we use \textit{Apache JMeter}\footnote{\url{http://jmeter.apache.org/}} which is a open source application designed for testing web services among others.
This tools allows us to simulate a typical workflow of using our REST API, while gathering various data about the performance and result of specific requests to the server.
Because \textit{JMeter} can be used to simulate the workflow, we are testing both the performance and the functional behaviour of the server.
Meaning that this test also is a system test -- And by increasing the amount of requests we send to the server in any given test, we can shift focus towards load testing.
Moreover \textit{JMeter}, can also distribute tests such that multiple computers can be used to load test the server.
By utilising this feature we can assure that the machine executing the test is not the bottleneck, and the outcome of any test will reflect the performance of the server.

\subsubsection{Test Plans}
Our first test plan aims to react all functional behaviour which is involved in a producer clients typical workflow.
This workflow can be summarized by the following points:
\begin{enumberate}
    \item Setup $n$ new test users.
    \begin{enumberate}
        \item Get authorization token for existing super user.
        \item Create $n$ new users and get authorization token for each.
    \end{enumberate}
    \item Create $m$ vehicles and save their ids.
    \item Create $x$ routes with random vehicle and random user.
    \item Each route has its own thread with creates $y$ waypoints, with a minimum delay of 1 second between each.
\end{enumberate}
\fxnote{Perhaps diagram instead?}

It should be noted that this test plan only submits data to the server, and advanced queries are not performed.
This means that the result only represents the server's ability to receive data.

\bigskip
To analyse the server's performance while executing queries, both simple and complex, and fetching data, we set up a test plan which does just that.

