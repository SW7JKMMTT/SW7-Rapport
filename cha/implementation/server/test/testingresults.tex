\subsection{Outcome of Internal Tests} % Kan ikke finde på en bedre titel.
In this subsection we will give some examples of where testing have proved useful,
i.e. concrete examples of tests which have revealed a bug or prevented a regression.

We have previously talked about the types of testing we have performed.
For the server the most essential are the integration tests,
during development they have caught a great number of mistakes and faults before deploying code.

% Waypoints sorteret i forkert rækkefølge
\subsubsection*{Ordering of Waypoints}
One of the mistakes we discovered by testing during development was bug which would order the waypoint in the wrong order when returning them in the \code{WaypointService}.

This bug was caused by a logical inconsistency in our code.
First when we defined our waypoints we defined them to be ordered by their timestamp in ascending order, as shown in \cref{lst:routewaypoint}.
This is done by setting the value of the \code{@OrderBy} property to ``timestamp'', which is ascending by default.
\begin{listing}
    \begin{java2}
@OneToMany(mappedBy = "route", fetch = FetchType.LAZY)
@OrderBy("timestamp")
private List<Waypoint> points;
    \end{java2}
    \caption{The property storing waypoints on the routes.}
    \label{lst:routewaypoint}
\end{listing}

However, our method to fetch waypoints by route, shown in \cref{lst:waypointdaobyroute}, with a maximum number of waypoints to return,
we had ordered it by timestamps descending.
This inconsistency was caught by an integration test, which tested that the waypoints on the route had the exact same content and ordering, as the waypoints returned by the \ac{DAO}. 
\begin{listing}
    \begin{java2}
public List<Waypoint> byRoute(Route route, int count) {
    TypedQuery<Waypoint> query = em.createQuery(
        "SELECT waypoint " +
            "FROM Waypoint waypoint " +
            "WHERE waypoint.route = :route " +
            "ORDER BY waypoint.timestamp DESC", Waypoint.class);
    query.setParameter("route", route);
    query.setMaxResults(count);
    return query.getResultList();
}
    \end{java2}
    \caption{The method in the \code{WaypointDao}, which fetches waypoints on a specific route with a given maximum.}
    \label{lst:waypointdaobyroute}
\end{listing}

% Join fetch giver duplicates.
\subsubsection*{Duplicates When Using Join Fetch}
We have previously described how we write \ac{HQL} queries to fetch from the database, in our \acp{DAO} in \cref{subsec:fetchfromdb}.
For example, if a \code{Vehicle} had two \code{Route}s, then fetching all \code{Vehicle}s, would yield said \code{Vehicle} twice.
When the \ac{DAO} fetched objects with collections, where the collections contained more than one element, the \ac{DAO} would return more than one of the objects.
This is because that when using a \code{JOIN} in \ac{HQL}, like in \ac{SQL}, the number of rows out is one for each \code{Route} in this case.
However we are only using this to fetch the object, Hibernate will put all \code{Route}s into the same collection on \code{Vehicle}.

This was unwanted, since one would not expect duplicates when getting all \code{Vehicle}s.
We discovered this with an integration test, and solved the problem by adding the predicate \code{DISTINCT} to the \ac{HQL} query, as we have on \cref{lst:userDaoGetAllForDisplay} for example.


% Hvis vi ønsker flere eksempler så kan vi bruge:
%   NoSQL -> SQL
%   waypont/datapoint add til lister

% Men de kræver noget kontekst at forklare ordenligt.
