\chapter{Testing the Server}\label{cha:testing}
In this chapter we will describe the following things:
\begin{itemize}
    \item The reason for testing.
    \item How much we should test.
    \item Our test methodology.
    \item Provide testing examples from the server.
\end{itemize}

Furthermore examples of how the testing have improved the server will be given.
We will also present some concrete examples of how these tests have found bugs and improved the server.

\section{Reasons for Testing}
The intention of testing software is to demonstrate that it does what it is intended to do.
According to Ian Sommerville in his book \textit{Software Engineering}\cite[p.~227]{software_engineering} testing is supposed to do two things:

\begin{description}
    \item[Validation testing]\cite[p.~227]{software_engineering} \hfill\\
    To test if the software meets its requirements validation testing is used.
    This means that there should, for a custom system, be a test for each requirement,
    demonstrating the ability of the software to fulfil the requirements.
    The test cases to test the requirements is supposed to reflect the intended use of the software,
    i.e. the tests should imitate regular use from a regular user.

    \item[Defect testing]\cite[p.~227]{software_engineering} \hfill\\
    To find input which makes the software behave incorrectly defect testing is used.
    The incorrect behaviour is the result of software defects,
    which can result in system crash, data corruption or other faults.
    Testing for incorrect behaviour also prevent regression during development,
    where new bugs could be introduced.
    Regression is the case where a piece of working software stops working after a new feature is added\cite{regression}.
    The test cases to expose the defects are not required to imitate regular use as validation testing.
    Therefore these test cases can be obscure, such that edge cases can be covered.
\end{description}

Since we are interested in the server being able to fulfil its requirements,
and we want to find and eliminate incorrect behaviour, we determine that we have to test the server.
Furthermore testing is a part of the requirements of the server, as mentioned in \cref{cha:requirements_elicitation}\fxnote{Add testing som requirement}.
Since our core area, as mentioned in \cref{sec:problem_statement} is the server, we will not test the front--end.
The next sections will describe the testing methods we intend to use.

\section{Types of Testing}
Ian Sommerville provides three stages of testing a system\cite[p.~231]{software_engineering}:
development testing, release testing, and user testing.
Development testing is tests made by the developers of the system during development
and includes unit testing, regression testing, component testing,
and system testing\cite[p.~232]{software_engineering}.

The second stage is release testing,
where a particular release of a system is prepared and tested to prove it is ready for use outside of the development environment\cite[p.~245]{software_engineering}.
Release testing includes requirements--based testing, scenario testing and performance testing.
The final test stage is user testing where users or customers provide feedback to the system.\cite[p.~249]{software_engineering}
User testing includes alpha testing, beta testing and acceptance testing.

User testing is deemed not to be relevant in testing the server,
since the server provides data for a front--end,
and an end user would only interact through a client with the server.
If we were to make a front--end, which is not only a proof of concept, user testing would be relevant.

We will perform development testing to test the server.
This choice is taken since we find elements from development testing relevant.
The methods we will use from development testing:
unit testing, regression testing, and component testing.

The form of release testing which Ian Sommerville describes do not fit very well with our development method.
This way of release testing is for major releases, where we instead create small and continuous releases.\fxnote{Vær sikker på at dette bliver nævnt tidligere.}
The way we will do release testing is by doing code review on code changes and make sure all development tests pass.
Nevertheless we will still do performance testing, not in the context of releasing, but to benchmark the server.

\bigskip

Another type of test we perform is smoke testing.
Smoke testing is an ad-hoc testing method where basic functionality is tested and happens naturally during development\cite{smoke_testing}.
An example of smoke testing in a software system is to make a compilation and see if it succeeds, since we use a statically typed language,
we can use the type system to check if all types are compatible.

To sum up, we will do unit testing, regression testing, and component testing which is a part of development testing.
Furthermore we will do our own version of release testing, do performance benchmarking,
and we will use smoke testing to quick test basic functionality.

\section{Code Coverage}\label{sec:code_coverage}
It is difficult to determine when a piece of software is adequately tested.

A way of measuring how much the code in a software project is tested is called code coverage.
There exist different metrics to measure code coverage, e.g. statement coverage and condition coverage,
and which one to use should be chosen according to the specific use case.
Performing a code coverage analysis can pinpoint untested areas of a software project.
It also calculates a quantitative measure of the code coverage,
which can be used to estimate if more tests are needed or not,
but it do not say what must be tested.\cite{code_coverage}

The downside of using code coverage as a goal for testing,
is that the code coverage measure of a software project does not say anything about the quality of the tests,
only that the code is being tested.
Therefore relying too heavily on having a high code coverage can give a false impression of the quality of the software.\cite{code_coverage_neg}

In a larger system having a code coverage of 100 \% is generally impractical and often it is not cost effective compared to the benefit gained by testing.
It is also estimated that a code coverage at 100 \% only expose about half the faults in a system.
Instead of testing everything the focus should be at finding bugs.
The point at which to stop testing is when the tests become contrived.\cite{code_coverage_not_100}

\bigskip

In regards to testing the server, we will not use code coverage as a goal for testing.
This choice is taken since code coverage do not give any qualitative information of the testing.
Instead code coverage will be used to find areas with low code coverage or no tests.
These areas will then be investigated to see if they need more tests or if the testing is adequate.

From the previously mentioned reason for testing,
it is stated that the components, which makes the software fulfil its requirements, should be tested.
Therefore when testing the server, the functionality,
which directly fulfil the requirements stated in \cref{cha:requirements_elicitation}, will be tested.

We do not deem it sufficient to only test the functionality which directly fulfil the requirements of the server.
Therefore selected parts of the server, which indirectly fulfil the requirements,
will also be tested to make sure they behave correctly.
The selected parts will be chosen from their importance in the functionality of the server,
for instance testing database connections,
which is not directly a part of the requirements from \cref{cha:requirements_elicitation}.

\fxnote{Dette skal revideres når de endelige krav kommer, f.eks. ryger det med databasen nok.
Alternativt kan det med indirekte krav slettes hvis det ``indirekte krav'' ikke anses som passende og så testen tilpasses så der kun testes for kravene.}

Thorough testing of the server will also help avoid regression during development.
Regression bugs can be hard to debug, since the system initially worked correctly and suddenly it does not.
With a combination of unit tests and integration tests together with the automatic running of tests,
the risk of not catching regression bugs is minimised\cite{regression}.

To avoid regression in the code, the execution of tests should be done automatically.
The reason for this is that a software projects can have as many good tests to test the code,
but if the developers forgets to run the tests during development,
the tests will be useless.

\bigskip
To summarize, we will test that the server satisfies the requirements,
and functions as intended.
Furthermore we will test other functionality which indirectly fulfil the requirements,
to avoid bugs and regression during development.
Finally the tests will be run automatically to make sure that regression bugs and other bugs are caught.

\section{Used Test Methods}
It is not possible to achieve an adequate level of testing of the server, using only one type of test.
This is caused by the fact that the server is made up of different components,
as mentioned in \cref{sec:internal_architecture},
which depends on and uses each other.
Furthermore testing the different aspects of the requirements, for instance ``Support load scalability.''
and ``Normalise incoming data and offer the data in a platform neutral format.'' puts completely different
requirements on the tests and testing methodology.
Therefore several testing methods is used to test the requirements to the server
and the important functionality to achieve an adequate level of testing.
The different testing methods we will use to test the server with is unit, integration, load, and system testing.

\subsection{Unit and integration testing}
For unit and integration testing, JUnit\footnote{\url{http://junit.org/junit4/}} with Hamcrest\footnote{\url{http://hamcrest.org/}} are used.
JUnit is a testing framework and Hamcrest is a matching framework, and therefore they have different types of assert methods.
Where JUnit have assert methods like \code{assertEquals}, Hamcrest have \code{assertThat}.
Since JUnit 4, Hamcrest's \code{assertThat} have been a part of JUnit and can be used without including Hamcrest separately\cite{hamcrest_vs_junit}.

The advantage of using \code{assertThat} over JUnit's assert methods is firstly the readability.
An example of the improved readability can be seen in \cref{lst:assert_readability},
where the \code{assertThat} statement at line 1 can be read as plain English,
but the \code{assertFalse} statement at line 3 can be harder to interpret for someone unknown to the code\cite{hamcrest_vs_junit}.

\begin{listing}
    \begin{java2}
        assertThat(actual, is(not(equalTo(expected))));

        assertFalse(expected.equals(actual));
    \end{java2}
    \caption{An example of the difference in readability taken from \cite{hamcrest_vs_junit}.}
    \label{lst:assert_readability}
\end{listing}

Furthermore the \code{assertThat} statement gives better failure messages\cite{hamcrest_vs_junit}.
An example of that can be seen in \cref{lst:assert_diff},
where a failed \code{assertTrue} returns the failure message at line 1 to 8 (only relevant information is showed),
which do not contain the values, which failed the test.
A failed \code{assertThat} returns the failure message at line 10 to 14,
but this message contains the values which failed the test,
which is helpful in debugging the reason the test failed.
It also makes it easier for people relatively unknown to the tests to debug a failed test.

\begin{listing}
    \begin{java2}
    junit.framework.AssertionFailedError
        at junit.framework.Assert.fail(Assert.java:55)
        at junit.framework.Assert.assertTrue(Assert.java:22)
        at junit.framework.Assert.assertTrue(Assert.java:31)
        at junit.framework.TestCase.assertTrue(TestCase.java:201)
        at rocks.stalin.sw708e16.server.persistence.TestSpatialRouteDao
            .testWithinRadius_WrappingAroundGlobeLongitude(TestSpatialRouteDao.java:225)
        ...

    java.lang.AssertionError:
    Expected: not a collection containing <rocks.stalin.sw708e16.server.core.spatial.Route@544e3900>
    but: was <[rocks.stalin.sw708e16.server.core.spatial.Route@544e3900]>
    Expected :not a collection containing <rocks.stalin.sw708e16.server.core.spatial.Route@544e3900>
    Actual   :<[rocks.stalin.sw708e16.server.core.spatial.Route@544e3900]>
    \end{java2}
    \caption{An example of the difference in failure messages.}
    \label{lst:assert_diff}
\end{listing}

To test the individual components of the server,
a combination of unit tests with JUnit and Hamcrest is used.
An individual component is a component which do not depend or interact with other components.

An example of a unit test from the server can be seen in \cref{lst:unit_test_example}.
The test checks if the user icons can be set on a user and retrieved.
First at line 4 a user is initialised without a user icon.
At line 7 a user icon is assigned to the created user,
and at line 10 it is tested that the user actually has the icon.
At line 1, the annotation \code{@Test} is JUnits way of defining a method as a test.

\begin{listing}
    \begin{java2}
        @Test
        public void testHasIcon_WithIcon_IsTrue() throws Exception {
            // Arrange
            User jeff = new User("Jeff", "passw0rd", "Jeff", "Lam");

            // Act
            jeff.setIcon(new UserIcon());

            // Assert
            assertThat(jeff.getHasIcon(), is(true));
        }
    \end{java2}
    \caption{An example of a unit test from \code{UserTest.java} which is a part of Core.}
    \label{lst:unit_test_example}
\end{listing}

As can be seen in \cref{lst:aaa_example}, we use the \ac{AAA} pattern when doing unit tests.
The \ac{AAA} pattern splits the tests up into three parts.
First is the Arrange part, where objects are initialised and variables declared and assigned.
Second is the Act part, where the invocation of the method which is tested is happening.
Lastly is the Assert part, where it is verified that the method which is tested is behaving as expected.\cite{aaa_pattern}
The test seen in \cref{lst:aaa_example} tests if a created user do not have a \code{authToken} as default.

\begin{listing}
    \begin{java2}
        @Test
        public void testListTokens_NoTokens() throws Exception {
            //Arrange
            User jeff = new GivenUser().withName("Jeff", "Jeffsen")
                .withUsername("Jeff").withPassword("password").in(userDao);

            // Act
            Collection<AuthToken> authTokens = authService.listTokens(jeff);

            // Assert
            Assert.assertTrue(authTokens.isEmpty());
        }
    \end{java2}
    \caption{An example of the use of \ac{AAA} pattern from \code{TestAuthenticationService.java} which is a part of Service.}
    \label{lst:aaa_example}
\end{listing}

\bigskip

When the components to be tested require more than them self to be tested,
for instance a database connection,
then integration testing is used.
The way the server is integration tested is the same way as unit tested,
except that the test depends on other components.
An example of this can be seen on \cref{lst:integration_test_example},
where at line 3 the \code{waypointDao} is used.
The \code{waypointDao} is used to communicate with the database,
and therefore this is an integration test instead of a unit test.

\begin{listing}
    \begin{java2}
        @Test
        public void testWithinRadius_WithNone_FindNone() throws Exception {
            List<Waypoint> ret = waypointDao.withinRadius(0, 0, 100);

            Assert.assertThat(ret, hasSize(0));
        }
    \end{java2}
    \caption{Integration test example from \code{WaypointDao.java} in Persistence.}
    \label{lst:integration_test_example}
\end{listing}

\subsection{The \code{@Before} annotation}
To make sure the state of the program is the same before each test and all resources needed are available,
the JUnit annotation \code{@Before} is used.
\code{@Before} is used to run methods before each test is executed,
for instance to set up needed objects\cite{before_doc}.
An example of the use of \code{@Before} in the server is to clear the test database,
to make sure the state of the database is the same before each test.
This is done by having all the test classes, which use the database, inherit the abstract class \code{DatabaseTest.java},
where the method \code{clearDatabase}, seen on \cref{lst:cleardatabase}, is defined.
The database is cleared before each test to make sure that each test has no assumptions about other tests,
since they are unordered and each should stand for itself.

\begin{listing}
    \begin{java2}
        @Before
        public void clearDatabase() {
            MongoClient client = new MongoClient(informationProvider.getUrl(), informationProvider.getPort());
            client.getDatabase("test").drop();
        }
    \end{java2}
    \caption{\code{clearDatabase} method from \code{DatabaseTest.java}.}
    \label{lst:cleardatabase}
\end{listing}

Another use of \code{@Before} in the server is to set up a waypoint service before there is made test on it.
The method \code{SetupWaypointService} from \code{TestWaypointService.java},
which can be seen in \cref{lst:setupwaypointservice},
creates a user, driver, vehicle, a route, and then sets the route.
The \code{setRoute} method at line 13 configures the \code{waypointService} to use the route \code{route}.
\code{TestWaypointService.java} also inherits \code{DatabaseTest.java},
so the way \code{@Before} works with multiple \code{@Before} statements in different classes
is that the \code{@Before} methods of the superclass will be executed before the \code{@Before} methods of the current class\cite{before_doc}.
This is convenient with the way we have constructed our tests since the newly inserted data will not be removed.

\bigskip

When test data is needed by the tests, test data is created and inserted in the database by using the \code{Given} classes.
An example of this can be seen in \cref{lst:setupwaypointservice}.
Using the \code{Given} classes gives a easy and transparent way of creating test data for the tests, which is also their only purpose.
By creating the \code{Given} classes as fluent interfaces\footnote{\url{http://martinfowler.com/bliki/FluentInterface.html}}, the method calls can be chained as shown in \cref{lst:setupwaypointservice} line 3 to 12,
thus the code can be read as plain English.

\begin{listing}
    \begin{java2}
        @Before
        public void SetupWaypointService() {
            User user = new GivenUser().withName("Jeff", "Jeffsen")
                .withUsername("Anders").withPassword("hunter2").in(userDao);
            Driver driver = new GivenDriver().withUser(user).in(driverDao);
            Vehicle vehicle = new GivenVehicle()
                .withMake("Ford")
                .withModel("Mondeo")
                .withVintage(1999)
                .withVin(new Vin("ABC123"))
                .in(vehicleDao);
            Route route = new GivenRoute().withDriver(driver).withVehicle(vehicle).in(routeDao);
            waypointService.setRoute(route);
        }
    \end{java2}
    \caption{\code{SetupWaypointService} method from \code{TestWaypointService.java} in Services.}
    \label{lst:setupwaypointservice}
\end{listing}

\subsection{Testing status}
As mentioned in \cref{sec:code_coverage}, we choose to test the requirements and important parts of the code.
%Måske en tabel som viser hvilke tests/classes som dækker hvilke requirements.
