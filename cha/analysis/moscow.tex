%!TEX root = ../../main.tex
\section{OBD PID prioritisation}
To prioritise all the different \ac{OBD} parameters, a MoSCoW analysis is performed.
A MoSCoW analysis is a prioritisation technique which prioritise into four categories:
``Must have'', ``Should have'', ``Could have'', and ``Won't have''.
The prioritization of the parameters is based on the the goals set in \fxnote{ref to goals},
e.g. if a goal is to be able to monitor the fuel economy of the vehicle,
then the \ac{OBD} parameters concerning the fuel economy is prioritized high.
\fxnote{Ret til at passe på et egentligt eksempel fra SOTA eller hvor kravene kommer til at stå.}
All the OBD parameters are from Wikipedia\cite{wiki-obd}
\fxnote{Håber jeg kan skaffe den orginale ISO standard, så vi ikke behøver Wikipedia.}.

On \myref{tab:moscow_pids} the ``Must have'' and ``Should have'' of the MoSCoW analysis can be seen.
Within each category there is no specific order, i.e. all within ``Must have'' have the same priority.
The full MoSCoW analysis including ``Could have'' and ``Won't have'' can be seen on \myref{tab:full_moscow_pids}.

``Must have'' contain parameters about the vehicle; Engine RPM, Vehicle speed etc., and parameters about the fuel consumption.
``Should have'' contains maintenance information and parameters which is not important enough to be ``Must have''.
``Could have'' contains the rest of the parameters.
Since the system have to store and provide data to clients, as much data as possible should be available for the clients.
Therefore the ``Won't have'' only contains the retrieval of pins not conforming to the OBD II standard\footnote{We don't have access to the original OBD PIDs list, only a list from 2002, which is heavily outdated.}
\fxnote{Som det er nu, så har jeg kun adgang til OBD standarden fra 2002, hvilket gør at der mangle mange pins og derved kan det godt være at de pins i won't have faktisk bliver brugt til noget (producent specifikt).}.

\begin{longtable}{|l|c|l|}
    \hline
    \multicolumn{3}{|c|}{\textbf{Must have}}                                \\ \hline
    \textbf{PID} & \textbf{Bytes returned} & \textbf{Description}           \\ \hline
    %Standard information
    00 & 4  & PIDs supported [01 - 20]                                      \\ \hline
    20 & 4  & PIDs supported [21 - 40]                                      \\ \hline
    40 & 4  & PIDs supported [41 - 60]                                      \\ \hline
    60 & 4  & PIDs supported [61 - 80]                                      \\ \hline
    0C & 2  & Engine RPM                                                    \\ \hline
    0D & 1  & Vehicle speed                                                 \\ \hline
    1C & 1  & OBD standards this vehicle conforms to                        \\ \hline
    1F & 2  & Run time since engine start                                   \\ \hline
    2F & 1  & Fuel Tank Level Input                                         \\ \hline
    51 & 1  & Fuel Type                                                     \\ \hline
    52 & 1  & Ethanol fuel \%                                               \\ \hline
    5B & 1  & Hybrid battery pack remaining life                            \\ \hline
    5C & 1  & Engine oil temperature                                        \\ \hline
    5F & 1  & Emission requirements to which vehicle is designed            \\ \hline
    61 & 1  & Driver's demand engine - percent torque                       \\ \hline
    62 & 1  & Actual engine - percent torque                                \\ \hline
    63 & 2  & Engine reference torque                                       \\ \hline
    64 & 5  & Engine percent torque data                                    \\ \hline
    7F & 13 & Engine run time                                               \\ \hline
    83 & 5  & NOx sensor                                                    \\ \hline
    %Fuel optimizing
    03 & 2  & Fuel system status                                            \\ \hline
    04 & 1  & Calculated engine load                                        \\ \hline
    0A & 1  & Fuel pressure (gauge pressure)                                \\ \hline
    11 & 1  & Throttle position                                             \\ \hline
    5A & 1  & Relative accelerator pedal position                           \\ \hline
    5E & 2  & Engine fuel rate                                              \\ \hline
    67 & 3  & Engine coolant temperature                                    \\ \hline
    \multicolumn{3}{|c|}{\textbf{Should have}}                              \\ \hline
    80 & 4  & PIDs supported [81 - A0]                                      \\ \hline
    %Maintenance information
    02 & 2  & Freeze DTC                                                    \\ \hline
    21 & 2  & Distance traveled with malfunction indicator lamp (MIL) on    \\ \hline
    30 & 1  & Warm-ups since codes cleared                                  \\ \hline
    31 & 2  & Distance traveled since codes cleared                         \\ \hline
    41 & 4  & Monitor status this drive cycle                               \\ \hline
    4D & 2  & Time run with MIL on                                          \\ \hline
    4E & 2  & Time since trouble codes cleared                              \\ \hline
    6F & 3  & Turbocharger compressor inlet pressure                        \\ \hline
    74 & 5  & Turbocharger RPM                                              \\ \hline
    \caption{MoSCoW analysis of the different OBD parameters.}
    \label{tab:moscow_pids}
\end{longtable}

The MoSCoW analysis will be used as requirements for which \ac{OBD} parameters the created system will support and store in the database.
\fxnote{Lidt mere rød tråd her, når der engang kommer en rækkefølge på analysis delen.}
