\chapter{Requirements Elicitation}\label{cha:requirements_elicitation}
As mentioned in \cref{sec:problem_statement} we wish to develop a back--end service capable of providing vehicle relevant information to a client, for all the vehicles they own, in whatever format they wish.
Before starting the design of the actual system we derive some requirements for the system.

First let us look at what makes up the system as a whole. \fxnote{Explain this better with rich pictures}
The system consists of three parts: a data provider, a back--end, and a front--end.
Data providers are relatively simple as they simply forward the data to the back--end.
The requirements for the front--end may differ significantly depending on the user and as such without a company as stakeholder would be a lot of guesswork to produce.
The data provider and back--end also depend somewhat on the user, but if all available data is collected, all data will be available to the front--end, whereas the front--end is very user specific and developing every possible UI and possible feature.
This leads us to focus on the back--end, the connection between front--end and data provider. \fxnote{argument not strong, any ideas? keep in mind architecture isnt yet decided(2 lines ahead)}
By focusing on the back--end we can also fit it for a $n..1..m$ structure.
In this structure the n represents the data providers, the 1 the back--end and the m the front--end clients.
This leads to us being able to have an arbitrary number of data providers and front--end clients communicate with the back--end.

It is important to keep in mind that all three constituents of the system can be isolated and considered as their own individual systems.
The data provider may be an app collecting and forwarding GPS and \ac{OBD} data to the back--end.
The front--end could be a system which requests data from the back--end.
The back--end could be a system that receives data from the data provider, normalises and stores the data, and forwards data when it is requested by the front--end.
Also note that data providers are linked to a front--end as they represent vehicles in the fleet and fleet owner, however no direct communication between the two occur.

\bigskip
The objective for the back--end is simply put to normalise, store and forward data to the front--end.
Vehicle relevant information can be acquired from \ac{OBD}, however this does not include geolocational data, as such an alternate way to provide geolocational data must be used.
All data must be collected and subsequently stored, such that it can be utilized by the front--end in whatever way the user desires.
Furthermore since vehicle and geolocational data will be retrieved from different sources, there must also be a method in place which links together information pertaining to the same vehicle.
Data should also be stored in sessions, such that data can be provided for a single driving session if so desired.

Developing a generic front--end that suits any company is not a realistic goal as no two companies are the same and may have different requirements to such a front--end.
Providing the data required allows a company to customize the front--end to their specific need.
This also means that the back--end must support providing data in a standardised platform neutral format, as to not impose any restrictions onto the front--end development.
Lastly \ac{OBD} is an evolving standard with the most recent addition being \ac{HDOBD}, as such the system should also be designed with this in mind; such that it can be further developed in order to support future \ac{OBD} standards, without the development causing issues for the rest of the system.

\bigskip
This  leads to the following functional requirements for the back--end:
\fxnote{convert to letters instead of bullet points}
\begin{itemize}
    \item Receive \ac{OBD} and GPS data from some data provider, such as an Android app.
    \item Access management for front--end requests.
    \item Normalise incoming data and offer the data in a platform neutral format. %Device scalability.
    \item Support load scalability.%Be load scalable such that an arbitrary number of vehicles and clients can make use of the service without new clients negatively impacting performance.
    \item Allow the front--end to request any data obtained from data providers.
    \item Support both functional and generational scalability.
\end{itemize}

%    \item Be able to read data from OBD and GPS off a mobile device.
%    \item Have a design which allows for easy scalability of the incoming data(OBD/HDOBD) and other possible extensions or future generations of OBD.
%    \item Normalize incoming data from their various format to a single format, this includes possible extension data from EOBD.
%    \item Provide data in various sensible formats such that a possible client isn't forced to use our idea of a ``proper'' format.
%    \item Provide any obtainable data to a possible client.
%    \item Acquire useful data from sources beyond OBD?? Fuel cost based on geolocation maybe?
%    \item Have scalability in regard to those measures defined earlier
%    \item next gen support, HDOBD most recent change to the OBD, more may come


%No outside stakeholders so we use SOTA and problemstatement, we are technically the stakeholders.
%We want to add new stuff so we provide data - data provided
%We want scalability - this is required to provide our services to a larger amount of clients ant to expand in the future
%Scalability is our key notion as our theoretical income is from clients using the system, for this to occur we want to support as many clients as possible

%Meta
    %what is req elicit, specify func reqs
%Core
    %System purpose?
    %Data reqs and purpose
    %Who is the user?
    %How does it help the user? -- tbh this should be evident in the intro as it should tell us why fleet management is a thing and what its use is?
    %So what it our job?
%Ending
    %The resulting requirements
