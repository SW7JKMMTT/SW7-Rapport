\chapter{Requirements Elicitation}\label{cha:requirements_elicitation}
As mentioned in \cref{sec:problem_statement} we wish to develop a system capable of providing vehicle relevant information to a possible user, for all the vehicles they own, in whatever format they wish.
Before starting the design of the actual system we derive some requirements for the system.

\section{Components of a Vehicle Management System}\label{sec:constituents_of_the_system}
First let us look at what makes up the system as a whole.
The system consists of three components: a data provider, a front--end, and a back--end.
\begin{description}
    \item [Data Providers] are the data collectors for the system. Their purpose is to collect data about the vehicle and its location and then forward that data to the back--end.
    \item [The Front--End] is the component that performs data analysis and represents this to an employee of the company using the system.
    Data analysis is a broad definition, the specifics of what this data analysis is would depend on the specific company, as different companies may require the front--end to perform different kinds of analysis.
    \item [The Back--End] This is the link between the data providers and front--end as they have no direct communication.
    The back--end is the storage component and also handles data validation such that no corrupt or incorrect data is stored.
    This component facilitates requests for the front--end such that it can use the stored data.
\end{description}

The aforementioned data analysis performed by the front--end may be different depending on the company's requirements, as such without a company as stakeholder it would be a lot of guesswork to develop and we therefore choose not to focus on the development of the front--end component.
The front--end will still be considered in regard to the other components of the system and a proof of concept solution will still be developed in order to show the interaction between the back--end and front--end.
Both the back--end and data provider also depend on the user, however this is to a lesser degree than the front--end.
For the data provider the user dependency is in obtaining data that is only available through manual input, or cooperation with other companies, e.g. fuel prices.
As for the back--end the dependency relies on what data should be available to the front--end.
Comparatively providing all available data may be time--consuming and thus may not be done, but has its perks as it opens up more features for the front--end, whereas developing every possible UI and possible feature is impossible.

\bigskip
Due to the back--end being the link between data providers and front--end clients it is arguably the most influential part of the system.
We have several data providers sending data to the back--end, the front--end clients may then request data from the back--end, provided by the data providers.
This means we have an arbitrary number of data providers and front--end clients all communicating with the same back--end.
A way to represent this would be $n..1..m$.
In this example the $n$ represents the data providers, the $1$ the back--end and the $m$ the front--end clients.
Not only can we have an arbitrary number of data providers and front--end clients, but they may also be different implementations, as long as they adhere to the communication protocol of the back--end.
Keeping in mind the aforementioned system interactions we deem the back--end the most influential part of the system, as such we choose to focus on developing the back--end.

\section{System Focus}
By choosing to focus on the back--end this also means that our users would be developers themselves, using our system as a data source to develop a front--end or data provider.
As such the system will need sufficient documentation such that a developer could develop such a system without knowledge of the internal specification and implementation of the back--end.
This documentation would be what the developers working with the clients should interact with, the documentation would fill the role of a user interface of sorts for the back--end.
The data provider may be developed by the same company as the front--end in order to achieve company specific specifications, or a generic data provider that forwards all available \ac{OBD} and GPS data could be created.
Whatever the choice may be the documentation should be the same for development of all types of clients.

It is important to keep in mind that all three components of the system can be isolated and considered as their own individual systems.
The data provider may be an app collecting and forwarding GPS and \ac{OBD} data to the back--end.
The front--end could be a system which requests data from the back--end, and visualises it through a browser.
The back--end could be a system that receives data from the data provider, potentially normalises and then stores the data, to be forwarded when it is requested by a front--end client.
Note that data providers are indirectly linked to a front--end as they represent vehicles in the fleet and fleet owner, however no direct communication between the two occur.
Furthermore it is paramount that the system acts on normalised data, such that misinterpretations can be avoided.

\bigskip
The objective for the back--end is simply put to store and forward data to the front--end.
Vehicle relevant information can be acquired from \ac{OBD}, however this does not include geospatial data, as such an alternate way to provide geospatial data must be used.
All data must be collected and subsequently stored, such that it can be utilised by the front--end in whatever way the user desires.
Furthermore since vehicle and geospatial data will be retrieved from different sources, there must also be a method in place which links together information pertaining to the same vehicle.
Data should also be stored in sessions, such that data can be provided for a single driving session if so desired.

Developing a generic front--end that suits any company is not a realistic goal as no two companies are the same and may have different requirements to such a front--end.
Providing the data required allows a company to customize the front--end to their specific need.
This also means that the back--end must support providing data in a standardised platform neutral format, as to not impose any restrictions onto the front--end development.
Lastly \ac{OBD} is an evolving standard with the most recent addition being \ac{HDOBD}, as such the system should also be designed with this in mind; such that it can be further developed in order to support future \ac{OBD} standards, without the development causing issues for the rest of the system.

\section{Requirements}\label{sec:requirements}
This leads to the following functional requirements for the back--end:
\begin{eletterate}
    \item Receive and store \ac{OBD} and GPS data from some data provider, such as a smartphone app.
    \item Access management for front--end requests.
    \item Offer normalised data in a platform neutral format. %Device scalability.
    \item Allow the front--end to request any data obtained from data providers.
\end{eletterate}
As well as the following non-functional requirements:
\begin{eletterate}[resume]
    \item Support load scalability.
    \item Support both functional and generational scalability.
    \item Be well--tested such that behaviour and performance is predictable.
    \item Be documented in regard to how data can be accessed and uploaded.
\end{eletterate}
