%Disp
%Meta stuff for archi design
    %Client server - very generally ending on a note of how it represents our architect
    %Internal architecture, server - our layered API stuff
    %Relating to requirements, how does the architecture fit our reqs? -- leads to scalability discussion of architecture
        %This discussion may contain elements of load scalability, moving data normalization to android app, modular design, rest.

%Client Server

%Each vehicle with phone = data provider
%Back-end collects data from multiple providers
%Back-end Provides data for x amounts of front-end's which a company may use

%We provide the Back-end
%We can assume that an android exists that can provide data(supervisor said so)
%May provide a proof of concept for Fron-end but that is not our purpose

%We provide DATA for business logic, the companies themselves will have to outsource another company to develop a front-end for them
%We provide PoC for front-end, theoretical business could have a baseline front-end but still provide data such that a company can outsource building them a specific application that suits their need using out data, that way the system they need developed is smaller, which is our business salespoint.

%Ours is client server, we then provide the database part of a layered system
%Our data collecntion part is essentialyl a reverse client-server architecture where multiple clients provide data to the server.

\chapter{Architectural Design}\label{cha:architectural_design}
As per the requirements elicitation, \cref{cha:requirements_elicitation}, the system we develop must be able to receive data from multiple devices, normalise and collect data and then provide a view of the normalised data to a number of clients.
In order to allow this, first an appropriate architecture that should be established as to avoid restructuring the system later on.

\section{Client--Server}
The most general way to describe the system we want to develop, would be that we want to provide a service which a number of clients make use of.
This very general description happens to fit with how Ian Sommerville describes the client--server pattern in \cite{software_engineering},
``[...] the Client--Server pattern is organized as a set of services and associated servers, and clients that access and use the services.''

As with any design choice the client--server architecture comes with both advantages and disadvantages, Ian Sommervile describes both advantages and disadvantages in \cite[p. 181]{software_engineering}.
The client--server architecture allows servers to be distributed across a network, something quite advantageous when considering load scalability.
It allows us to control the flow of data to each server by controlling which servers specific clients communicate with.
Another advantage of the client--server architecture is that clients may use some functionality provided by the server and as such have no need to implement those functions.
These advantages are a result of the key benefit in the client--server architecture, separation and independence, allowing clients and servers to be changed without affecting each other as communication is handled through an abstraction layer, for our case such a layer may be an API.

As mentioned the client--server architecture comes with both advantages and disadvantages.
Ian Sommerville mentions disadvantages as management in regard to servers owned by different organizations, performance due to network and security in regard to how vulnerable the system is to denial--of--service attacks.
As for management issues in regard to servers, a way to handle this is to implement some form of access management such that requests are designated appropriately.
As for denial--of--service attacks there is no way to completely secure oneself from these attacks, although some preventive meassures in regard to load balancing and request limits can be used.

\bigskip
A key point for the client--server architecture is its inherintly distributed structure, a structure we also require if we are to be able to have multiple front--end clients and data providers.
The client--server is also a well defined and broadly used architecture by large--scale web services such as Netflix who have adapted it to suit their specific needs and work with their tenchnology stack which can be seen on their techblog \cite{netflix}.
For our particular system we have two types of distinct clients communicating with the server, as represented on \cref{fig:client_server_architecture}.
\begin{description}
    \item [Producer] which are the clients providing data, these are located in the vehicle and transmit both \ac{OBD} and GPS data to the server.
    \item [Consumer] which are the clients making use of the data provided by the producer, this is the system that a company would use, these request data from the server.
\end{description}
\subimport{}{interwebz_architecture.tex}
\fxnote{figure needs fixing}

\section{Internal Architecture}\label{sec:internal_architecture}
While the client--server pattern is used for the system as a whole; producer, server, and consumer; each of these three components may have their own internal architecture.
Designing and implementing all three constituents of the system would require signifcant work, more so than we have time for as such we choose to not implement the data provider, although its effect on the system as a whole is still considered.
The data provider can be emulated without any issues as it simply needs to forward data.
A single component does not know the structure of the other components, they simply know of their existence, as such emulating the provider rather than developing it does not affect the back--end or front--end directly.
Choosing to emulate does remove some options for load scalability, these are covered in \cref{sec:data_normalisation}.

\subsection{Server}
For the server component we use a layered architecture consisting of three layers.
\begin{description}
    \item [Core] This layer contains both data model and business logic, an UML diagram for this layer is shown in \cref{fig:model_class_diagram}.
    \item [Persistence] This layer contains \aclp{DAO}.
    \acsp{DAO} determine how an object may be accessed and what data operations are allowed on the data, protecting the database details.
    They are further explained in \cref{subsubsec:dao}.
    \item [Service] This layer contains the code which handles requests.
    The service layer is the layer which the clients communicate with when they wish to make requests to the application.
    It contains end--points for each available request, if the request from a client is legitimate, the endpoint makes use of the \acsp{DAO} to manipulate or retrieve data.
\end{description}
As mentioned this helps hide server details that neither client needs to know about.
The only layer that clients interact with directly is the service layer, having the layers be separate also helps in case the model needs altering for some reason, as only the core would have to be refined.
The layered architecture also brings in the concept of functional scalability, as all data is present, only the service layer would be altered in order to provide the clients with new functionalities, given that the data is available in the system.
How this further helps scalability is described in more detail in the implementation section \fxnote{opdater når denne section eksistere}.

%\subsubsection{Client--Producer}
%Depends on user interaction wished for/wanted --- if low to none then pipe and filter layer is sensible
%If high wished, then a layered/MVC Pattern may be wanted
\subsection{Consumer--Client}
The data provided by the system we are developing makes it such that any company can have a system built that suits their exact needs; this also means that the architecture for said system will be whatever they need.
As an example a consumer client could use a layered architecture as shown on \cref{fig:consumer_client}.\fxnote{Kasper:Design figure to you implementation and rewrite this accordingly}
A generic layered architecure might have the four following layers, as described by Ian Sommerville in \cite{software_engineering}:
\begin{itemize}
    \item User Interface.
    \item User interface management, Authentication and authorization.
    \item Core business logic/application functionality, System utilities.
    \item System support (OS, database, etc.)
\end{itemize}
\subimport{}{consumer_client_fig.tex}
For a consumer client to communicate with the server, the fourth layer may be a dedicated layer to do so as represented on \cref{fig:consumer_client}.
Alternatively there could be a fifth layer that does that, and then stores data locally to reduce the number of requests.
For the consumer, essentially any arbitrary architecture could be chosen as long as it supports communicating with the server; the choice would depend on the specific use case and needs of the client, \cref{fig:consumer_client} is merely an example for how we have developed our proof of concept front--end.

\subimport{}{data_norm.tex}

\section{Scalability Concerns}
With scalability being a central requirement for our system, we evaluate how our choice of architecture affects this.
First we consider version and device scalability.
Both of these scalability measures are affected by the communication between client and server rather than their internal architectures.
In order to attain device scalability the data format must be neutral.
The client--server pattern only concerns itself with the seperation of client and server leaving the communication between the two up to the developer.
Due to the choice of communication being our decision, simply deciding to use a device neutral format such as JSON will make the system support device scalability.
As for version scalability this is only relevant for the producer client and server as the consumer client has no need to know what version of \ac{OBD} is used.
In our case version scalability is tied heavily to functional scalability, as for the server it would be expanding what information it can receive.
As for functional scalability the choice of client--server architecture does not affect this, as this is acquired by having low coupling within the server component.
This results in both version and functional scalability relying on the server's internal architecture, which happens to be a layered architecure which faciliates low coupling and through low coupling functional scalability.

The last scalability measure we have established is load scalability, for this measure the client--server architecture allows the client to communicate so some server, without knowing which one; thereby allowing us to introduce horizontal scalability.
