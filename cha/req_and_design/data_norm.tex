%!TEX root = ../../main.tex
\section{Data normalisation}\label{sec:data_normalisation}
As seen in \cref{sec:architectural_design} the data from the vehicles is provided by a generic data provider.
This generic data provider can be a arbitrary piece of equipment, for instance a smartphone. 
Since it is different types of vehicles the data is extracted from, the data format can differ.
Therefore the data needs to be normalised before the back--end can properly use the data.
The process of normalising the data is to unify the format, which could differ in units, have a lack of some data or the data could be corrupted.
An advantage of normalising the data is that it does not matter where the data is from when it have been normalised, since the data have a unified format.

The way the system is split up, as seen in \cref{sec:architectural_design}, there are two options of normalising the data: centralised or distributed.

\begin{description}
    \item[Centralised] \hfill \\
    A centralised normalisation of the data means that the data is normalised at a central point, in our case on the back--end.
    The data provider sends the raw data and before the back--end can use the data it has to normalise it.
    If we were to choose this centralised model, it would mean that no matter if there are 100 vehicles or 100.000 vehicles, 
    the back--end is responsible of normalising all the sent data.
    
    \item[Distributed] \hfill \\
    With a distributed normalisation, the data is normalised at different places, in our case at the data provider. 
    This means that when the back--end receives the data it has been normalized and is ready to be stored.
\end{description}
\bigskip

Both ways of normalising the data is possible to do in our system.
A disadvantage of a centralised normalisation is a extra workload on the back--end.
The extra workload will probably not be a problem with a small amount of vehicles sending data.
In the case of many vehicles, for instance 100.000, sending simultaneously, the back--end could have problems with handling all of the data,
and its other tasks, thus creating a bottleneck in the system.
This bottleneck will impair the load scalability of the system, which is defined in \cref{subsec:scalability},
since it can have negative effect on how many data-providers and clients that can communicate with the back--end.

This problem does not occur with a distributed normalisation, since the data provider in each vehicle deals with the normalisation.
The data provider in the vehicle only deals with relaying the data from the vehicle, 
so we expect that a data provider will have no issues with also normalising the data from the vehicle it belongs to.

Another advantage of a distributed system is that the interface of the back--end can be kept clean.
The server interface will not need to have a lot of edge cases, 
these will be kept at the individual vehicle that needs them, 
thus giving the system, in regard to normalising the data, a more logical structure.
Keeping the edge cases at the vehicles and keeping the back--end interface clean improves also the device scalability, 
as mentioned in \cref{subsec:scalability}.
In a centralised system, the edge cases can risk creating conflicts, but that is eliminated in a distributed system,
thus improving the device scalability and making it easier to extend support for different types of vehicles.

\bigskip

From the arguments stated above we choose to distribute the data normalization between the data provider and back--end.
Since the data provider already is in the vehicles, 
we believe it gives a better distribution of the computational requirements of the whole system.
Furthermore since we opt for scalability in this paper, the improvement of the device scalability is valued high.
