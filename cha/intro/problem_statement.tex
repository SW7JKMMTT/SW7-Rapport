\section{Problem Statement}\label{sec:problem_statement}

The focus of this project will be on designing and implementing a web service,
which can act as the back--end for various vehicle fleet management solutions.
The architectural design of the aforementioned back--end will include components
from data gathering devices resident in individual vehicles,
to the central system which aggregates and stores the data.

The ability to handle live data will be a key aspect of the system,
where live data reflects the current state of objects being tracked
--- henceforth we will refer to such data as \textit{timely}.
Moreover a proof of concept front--end client will be made,
to display certain aspects of the back--ends capabilities.
This front--end client, will be a web application targeted at a potential user.

Since the handling of timely data is a key aspect of the system,
it should be able to, in all cases, keep up with the demand put by the vehicles and users.
This means that the system should have the ability to scale accordingly to the requirements put on it.
Therefore the scalability of the system is an important part of this paper.

\bigskip\noindent
This is expressed by the following problem statement:

\medskip
{\addtolength{\leftskip}{10mm}\addtolength{\rightskip}{10mm}\noindent\hrulefill\it

\noindent How can a scalable system be designed and implemented,
such that geospatial and relevant data can be collected and aggregated in a timely way from a disjoint fleet of vehicles,
while maintaining arbitrary possibilities of presenting the data to a given user?

\noindent\hrulefill

}
