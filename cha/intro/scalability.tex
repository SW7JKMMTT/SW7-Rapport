\subsection{Scalability}
While definitions of scalability are easy enough to find, they differ slightly in their definitions of scalability, the following four definitions show this:

The first two examples are from the same online source and defines scalability as both;
``It is the ability of a computer application or product (hardware or software) to continue to function well when it (or its context) is changed in size or volume in order to meet a user[cic] need.''
and;

``It is the ability not only to function well in the rescaled situation, but to actually take full advantage of it.''
\cite{scaleDef1}

The third example is from a computer scientist's blog who defines it as;
``It is used to indicate a hardware design that allows the system to be increased in size and in doing so to obtain increased performance. This could be described as architecture or hardware scalability. Scalability is also used to indicate that a parallel algorithm can accommodate increased data items with a low and bounded increase in computational steps.''
\cite{scaleDef2}

The fourth example is from the introduction of a paper with the purpose of defining scalability;
``Scalability is a desirable attribute of a network, system, or
process. The concept connotes the ability of a system to
accommodate an increasing number of elements or objects, to
process growing volumes of work gracefully, and/or to be
susceptible to enlargement.''
\cite{scaleDef3}

Despite the differences in definition, they all converge on the same idea; a system designed with expansion in mind, leading to expansion becoming a simple task rather than a complicated one.
As such the following list will define different scalability measures, relevant in order to solve the problem presented in \cref{sec:problem_statement}.
The following definitions will be how scalability is referenced henceforth throughout this paper.
\begin{description}
    \item [Functional Scalability]
        referring to the functionality available to the user.
        Scalability in this regard refers to the effort required to add new functionality.
    \item [Load Scalability]
        referring to the capability of handling increasing amounts of data and requests.
        In load scalability one can make a distinction between two kinds of scalability as described in \cite{HoriVertScale}.
        \begin{description}
            \item [Horizontal] also known as ``scale--out'', which refers to increasing the ability to handle requests and data by increasing the number of machines in use, i.e. setting up clusters rather than a single machine.
            \item [Vertical] also known as ``scale--out'', which refers to increasing the capability of the single machine through adding processing power, storage etc.
        \end{description}
    \item [Generation Scalability]
        referring to the ability to support different generations, particularly new generations.
        This helps when adapting the system to the new generation in order to support new features added, regardless of what this new thing may be; this relates closely to attaining low coupling between the system modules.
        For our particular project a generation would refer to the possible expansion of \ac{OBD} as well as the different generations already in use mentioned in \cref{sec:on_board_diagnostics}.
    \item [Device Scalability]
        is similar to generation scalability, but targeted at supporting a variety of devices, rather than generations for a single system or device.
\end{description}
