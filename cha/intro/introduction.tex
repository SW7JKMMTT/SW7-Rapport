\chapter{Introduction}\label{cha:introduction}
Vehicles are an intricate part of society today, both in private and corporate regard.
In 2015, 60~\% of all Danish families had a car while 15~\% had two or more\cite{MoreCarsInFamilies}.
In the corporate world, corporations may provide their employees with vehicles,
and some even require that you own a car to perform the duties that follows the job.
Some companies utilise vehicles as a significant component,
and their business model is dependent on the availability of said vehicles.
These are companies such as shipping firms, transportation services,
as well as government institutions e.g.~police departments and the military.

\bigskip
For a Danish family managing one or two cars is a simple task,
in comparison to companies with fleets of vehicles where simply acquiring an overview can be complex.
Moreover these fleets can contain several hundred disjoint vehicles;
disjoint meaning vehicles of varying types, e.g.~cars, trucks, vans, but also make and model.
This only contributes to the complexity of managing the fleet.
The domain of live monitoring larger vehicles such as planes and ships is already well explored and covered
due to the potential catastrophic consequences of a failure;
because of this we will focus on road vehicles in this project.
Between the late 1990s to early 2000s, the possibility of monitoring road vehicles through the use of \acl{OBD},
was made mandatory in all vehicles in the European Union by an EU directive\cite{EUDirective}.

The \ac{OBD} system allows retrieval of various data about the car, such as fuel pressure, speed, engine status etc.
A collection of this data can then be utilised by companies to monitor and manage their fleet of vehicles.
Moreover GPS and other additional data can be combined with the output of the \ac{OBD} system
to provide significantly more information in the management system.

\bigskip
For a fleet management system to enable proactive choices about managing the fleet,
the system must present data in a timely way, i.e.~before a preventable event happens.
Moreover different companies have different requirements for what the ideal fleet management system can do,
e.g.~shipping firms may need parcel information alongside different vehicles,
and public transit companies might benefit from statistics about the behaviour of their passengers.
Because of such different requirements, customisation of a fleet management system,
will enable it to suit any use case, thus improving adaptation and usefulness of the system.

