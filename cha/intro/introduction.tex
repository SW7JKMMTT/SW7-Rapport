\chapter{Introduction}
Vehicles are an intricate part of society today, both in private and corporate regard.
In the western world having a car is the norm, and for couples or families having two cars is far from uncommon.
Relying this heavily on vehicles extends far beyond the needs of the common man.
In the corporate world, corporations may provide their employees with vehicles, and some even require that you own a car as it is required to perform the duties that follows with the job.
Some companies rely far more on vehicles than others such as shipping firms, transportation services as well as the government in regards to police cars, ambulances etc.

For the common man managing one or two cars is a simple task, however when that is scaled up companies having fleets containing several hundreds vehicles of varying types, cars, trucks, vans etc., acquiring an overview and information for each vehicle is no longer a simple task.
Where as larger vehicles such as planes and ships are closely monitored due to the catastrophic consequences of a fault, road vehicles are typically not as well monitored.
Between the late 90s to early 2000s at least providing the opportunity to monitor road vehicles was made mandatory in the western world through the use of \acl{OBD}.\cite{EUDirective}

The \ac{OBD} system allows us to retrieve various data about the car and its current state, combined with GPS this allows for data collection that can be utilized to provide companies the opportunity to monitor and manage their fleet of vehicles.
