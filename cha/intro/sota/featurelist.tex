%!TEX root=../../../main.tex
\subsection{Featurelist}
%Formetly known as Analysis and synth chap, may need another name, also the purpose for this is still very unclear, maybe it should just be removed?.
Even though \textit{Fleetio} manages to be able to use the more advanced hardware such as \textit{Geotab}'s Plug-\&-Play OBD--II device, it does not take full advantage of it.
When \textit{Fleetio} uses pre--existing hardware, it takes gathered data and uses it to substantiate already existing data such as the current speed, which e.g. \ac{OBD} can measure more precisely.
\textit{Fleetio} also require manual input of several different data points, such as last refuelling.

In order to improve upon the pre--existing platform which \textit{Fleetio} has already developed, the support of third party modules such as \ac{OBD} modules, could improve the platform.
By looking at the already existing solutions the reuse of some functionality might be beneficial, some functionality could be the following:
\begin{description}
    \item[Vehicle profiles] \hfill \\
    The vehicle profiles are used to keep track of the documents and information which are associated with each vehicle in a company.
    \item[Maintenance history] \hfill \\
    The document of the maintenance history and which repairs that have been performed on each vehicle.
    \item[Employee profiles] \hfill \\
    An overview of each of the employees, and the ability to store specific information, e.g. the driving behaviour of a driver.
    \item[Real--time cost per kilometre] \hfill \\
    The tracking of the different vehicles position, and the amount of petrol in the tank allows for calculating the overall fuel usage, and if the price is given, an overall cost can be calculated.
\end{description}

However, some functionality might not apply to all companies within the scope of a similar system, some of these functionalities could be the following:
\begin{description}
    \item[Vendor overview] \hfill \\
    In a similar solution it might not be deemed as necessary to keep track of which vendors are used, since it is more important to keep track of the overall expense rather then the places where money is spend.
    \item[Driver scheduling] \hfill \\
    The scheduling of the routes is not relevant for this project, since the focus is on the tracking of vehicles and the properties of the vehicles.
\end{description}

\section{Summary}\label{sub:Assessment}
In order to implement a fleet management system, a reliable back--end which will be able to support some of the features that were deemed beneficial for a similar system is important.
This leads us to focus on the missing aspects of the pre--existing solution \textit{Fleetio}, which is live vehicle information whichroutescan be extracted through \ac{OBD}, this can help improve automatization, and help predict maintenance.
