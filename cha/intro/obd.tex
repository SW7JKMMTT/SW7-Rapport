\section{On--Board Diagnostics}\label{sec:on_board_diagnostics}
\ac{OBD} refers to the computer in a vehicle that provides information about the vehicle;
ranging from simpler information such as kilometers driven to more complex real--time mechanical factors such as RPM, speed, airflow rate etc.
Since the origin of \ac{OBD} in the early 80's by General Motors, who referred to it as Assembly Line Diagnostic Link, it has undergone generations of development before reaching a standard, more information on ALDL can be found at \cite{ALDL}.

According to \cite{CaliforniaEPA} OBD--I was introduced as a standard in California in 1988, this eventuelly lead to the OBD--II standard used today.
The standard specifies the diagnostic connector and its pin-out, signalling protocols and messaging format alongside a list of vehicle parameters to monitor and how to encode the data for each parameter.

In 1996 this standard was made mandatory for all cars manufactured in the US.
Since OBD--II became a requirement in the US it has since been adopted in other regions of the world, each using their own slight deviation of the standard.
The following lists contains those deviations the more relevant ones described in the following list:

\begin{description}
    \item[EOBD]\hfill\\
        A European equivalent to OBD--II, this standard was introduced in Europe in 2001, for more information on the EU Directive in relation to EOBD can be found at \cite{EUDirective}.
    \item[EOBD2]\hfill\\
        Unrelated to EOBD, EOBD2 is a marketing term used by manufacturers to refer to their Enhanced standard providing features beyond what OBD--II and EOBD requires.
    \item[JOBD]\hfill\\
        The Japanese equivalent of OBD--II
    \item[ADR 79/01 \& 79/02]\hfill\\
        The Australian equivalent of OBD--II and also the latest regulation implemented in 2006, more information on this regulation can be found at \cite{AustralianRegulation}.
    \item[HDOBD]\hfill\\
        \ac{HDOBD} is the standard for \ac{OBD} in trucks and heavy--duty vehicles.
        The main difference between \ac{HDOBD} and OBD--II is extra emission monitoring\cite{hdobd}.
\end{description}

\bigskip
Despite \ac{OBD} being a standard, it does not have a universal signal protocol, instead it has the following five:
\begin{multicols}{2}
\begin{itemize}
    \item SAE J1850 PWM
    \item SAE J1850 VPW
    \item ISO 9141-2
    \item ISO 14230 KWP2000
    \item ISO 15765 CAN
\end{itemize}
\end{multicols}
Which protocol is used for the specific vehicle is commonly deducible by what pins are present in the connector.

The information collected by the \ac{ECU} is available for use through the OBD--II port.
This information can combined with the \ac{DTC} assist in finding issues with the engine before a critical problem arises.
Hence this information could be used to resolve problems before affecting the user.

\medskip
It is worth noting that \ac{OBD} is not a very secure tool; researchers at the University of Washington and the University of California have shown that it is possible to gain control of a vehicle by uploading a new \ac{ECU} firmware\cite{OBDSecurity}.
