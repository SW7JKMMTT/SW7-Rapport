\section{Use Case}
\acl{VFM} has a wide variety of potential use cases all with both unique aspects, as well as some shared ones.
In order to work more focused on the problem a single use case is chosen as the primary focus.

The following list presents some of the use cases discussed, as well the aspects that were not shared between all the cases:

\begin{description}
    \item [Taxi/Uber] \hfill \\
    The information from \ac{OBD} could be used in order to provide customers with more specific information in regards to pickup time.
    Another aspect here is customer service, through \ac{OBD} the drivers can be evaluated in regard to breaking patterns, speed etc.
    \item [Military (Logistics division)] \hfill \\
    In the military a variety of vehicles are used, each with unique aspects to them as most of the vehicles differ from civil vehicles in some fashion.
    Furthermore vehicles in the military are often away from the base days at a time, at which point having information about the vehicles status and position becomes even more convenient than normal.

    When training there are also regulations depending on what type of vehicle that must be followed, the information from \ac{OBD} would help to reveal if the regulations are upheld.

    Working with military vehicles would impose an additional requirement to consider the security of \ac{OBD}, both in order to encrypt the data, but also security measures in order to avoid the potential hijacking of a vehicle through the \ac{OBD}.
    \item [Lorry Drivers] \hfill \\
    Similar to the military trucks, lorry drivers are often on the road for days at a time when driving a cargo truck.
    Lorry drivers also have regulations about sleep, cargo weight etc. which \ac{OBD} could provide data for, in order for the truck company to check whether their drivers uphold those regulations.
    \item [Postal Company (Post Nord)] \hfill \\
    For the postal company the primary goal is to optimise the route their vehicles take, and to use all vehicles efficiently to cover the area where packages must be delivered.
    \fxnote{Jeg har fuldstændig glemt hvad det unikke aspect var her}
    \item [City Buses (Nordjysk Transport)] \hfill \\
    City buses run on a time schedule and those who use public transit expect the buses to be on time, information retrieved from the \ac{OBD} could help reveal potential delays, as well as where the delays occur such that a solution can be found.
    Similar to taxi and Uber drivers customer service in regards to the performance of the driver is also an important aspect for bus drivers.
\end{description}

\bigskip
Aside from these unique aspects, one shared aspect affects all cases --- maintenance.
\ac{OBD} provides data that can reveal potential maintenance issues and risks ahead of time, a concern relevant to all professions that rely on vehicles.
However, vehicles used for eight hour shifts are usually in the garage when not in use, or at least close to one, this makes the maintenance aspect less critical for those use cases.
As for both military vehicles and cargo trucks this is not necessarily true as they often are gone for months at a time.

Although, due to the variety of military vehicle, cargo trucks are chosen at the focus of this paper.
Decisions made henceforth will be made accordingly to the focus being on cargo trucks; this should lead to an architecture with a shared basis expanded to consider cargo truck specific aspects.
The intention is that the expanded part is easily modifiable in order to target other types of vehicles and their unique attributes.

The driver of a truck is called a ``hauler'' and is often employed by a haulage contractor, which is a company that administer the trips by the hauler and owns the trucks.
In Denmark a hauler is required to follow special regulations concerning driving and resting times\cite{haulierSleepLaw}.
Henceforth in this paper, when the word ``hauler'' or ``haulage contractor'' is used, it refers to a truck hauler and a truck haulage contractor. 
