%!TEX root = ../main.tex
\section{State of the art}
For us to determine the state of the art of the fleet management systems, six different fleet management systems have been investigated.
Three from ``THE TOP 20 MOST POPULAR Fleet Management Software''\cite{top20fleetmanagement} and the top three most popular results from an internet search.
The chosen products were: TomTom WEBFLEET\cite{tomtom}, Teletrack Navman DIRECTOR Fleet Software\cite{teletracnavman}, Geotab MyGeotab\cite{geotab}, Enterprise Fleet Management\cite{efleets}, Vinitysoft Fleet Management Software 4.0\cite{vinitysoft} and Fleetio\cite{fleetio}. 
We choose these products to give a good representation of the most used vehicle fleet management systems, since three of them was top three at a rating site and the other three was the top three most popular relevant search results.
Therefore these products are expected to be a representation of the state of the art.

\subsection*{Technologies}
From the investigation of the different fleet management systems, a list of requirements was created.
These chosen requirements are features which was found essential to create a state of the art fleet management system.

\begin{description}
    \item[Vehicle registration] \hfill \\
    It covers the registration of the vehicle in the system, e.g. make and model, registration papers, \ac{VIN}, driver of the vehicle etc.
    \item[Fuel management] \hfill \\
    Enabling the registration and aggregation of statistics regarding the fuel consumption of a given vehicle.     
    \item[Insurance and accident] \hfill \\
    Makes it possible to track accidents and the insurance history of a vehicle. In the case of accidents it can help register and classify the damage and the insurance status of the vehicle. 
    \item[Position tracking] \hfill \\
    Enables tracking which can give a precise and real-time position of the vehicle. 
    \item[\ac{OBD} support] \hfill \\
    It covers the ability of using the data provided by the \ac{OBD} port of a car. \ac{OBD} gives a variety of data from the car, for instance engine RMP, throttle position, intake air temperature and a lot of other parameters\cite{wiki_obd}. 
    \item[Maintenance tracking] \hfill \\
    Helps keep track of a vehicle's maintenance schedule and history. It should be able to use \ac{OBD} data to analyse if a car needs maintenance outside of the regular schedule and to make a more dynamic and individual maintenance schedule.
    \item[Work order tracking] \hfill \\
    It manages and keeps track of the work progress of the vehicle. In the case of tracking a delivery truck, it would keep track of the deliveries.
    \item[Web/mobile UI] \hfill \\
    It gives the user the possibility of accessing the fleet management system from a web browser or on a portable device. 
\end{description}

This list of features is what a fleet vehicle management system is required to be able to do to meet the state of the art.
Therefore these features will be used as a foundation for the requirements, for what the backend service will be able to provide.

\fxnote{Indsæt noget rød tråd her, noget ref til næste sektion/kapitel og nævn hvad det skal bruges til. Er lidt usikker på hvor i rapporten denne sektion skal være, derfor jeg laver denne note og indsætter det når jeg har lidt bedre kontekst på det.}