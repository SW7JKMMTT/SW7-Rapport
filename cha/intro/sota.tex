%!TEX root = ../../main.tex
This chapter grants an overview of some existing fleet management systems, then data gathered and the state of the art is analysed.

\subsection{Knowledge}\label{sub:Knowledge}
In order to determine the current sate of the art systems, the focus falls upon a review site where a comparison between each of the different systems is available\cite{top20fleetmanagement}.
The comparison includes the following solutions:
\begin{description}
    \item [\textit{Fleet Management Software 4.0}]\cite{vinitysoft}: The \textit{Fleet Management Software 4.0} is developed by Vinity Soft, and focuses on keeping track of vehicle expenses, creating maintenance programs and service schedules.
    \item [\textit{TomTom Webfleet}]\cite{tomtom}: The \textit{TomTom Webfleet} is developed by the Dutch company TomTom. TomTom is known for their GPS navigation systems. This solution uses the TomTom LINK device to track vehicles' position and movement, and transmits this information to the \textit{Webfleet} dashboard. TomTom also provide anAPI for integration with existing systems.
    \item [\textit{TELETRAC NAVMAN DIRECTOR}]\cite{teletracnavman}: The \textit{TELETRAC NAVMAN DIRECTOR} solution is according to Teletrac, \enquote{the future of fleet management}, some of the tools that \textit{Navman Director} provides are tools for fleet \& asset management, driver behaviour \& safety.
    \item [\textit{Geotab}]\cite{geotab}: The \textit{Geotab} solution uses a device which is to be mounted in the vehicles in order to track them. MyGeotab, which is the online dashboard for \textit{Geotab}, is able to track the position of the different vehicles it is also able to detect vehicle accidents.
    \textit{Geotab} also allow for maintenance reminders by time or distance.
    \item [\textit{Fleetio}]\cite{fleetio}: The \textit{Fleetio} solution takes advantage of the drivers phone, instead of requiring a separate device. Some of the areas which \textit{Fleetio} focuses on is managing misc data, keeping track of fleet maintenance and increasing the effectiveness of a company.
\end{description}
Each of the different fleet management systems allow for either manual or automatic tracking of the daily use of the different vehicles.
Most of the systems also allow for tracking of the vehicles through GPS and advanced vehicle information, such as price per kilometre and driving behaviors.

\subsection{Understanding the gained knowledge}\label{sub:understanding}
The information gathered in the previous section is used to attain a deeper understanding of the the different fleet management systems.

\textit{TomTom Webfleet} and \textit{TELETRAC NAVMAN DIRECTOR}, both mainly focus on the driver aspects of fleet management.
Some similarities between the two are:
\begin{description}
    \item Overview of driving behaviour.
    \item Calculating the expenses for each of the drivers.
    \item Live overview of the fleet via GPS.
    \item Keeping track of the delivery time.
\end{description}

\textit{Fleet Management Software 4.0} takes a rather different approach, this solution allows for manual tracking, and the ability to store documents of the different vehicles.
Some of the other tools which is available is the ability to schedule vehicles for maintenance and a limited way of calculating the expense of the drivers this is done by manually adding each refuelling and expense to the system.

\textit{Geotab} takes a different approach, in order for this system to work a Plug-\&-Play OBD--II device has to be mounted into each of the vehicles, the device then transmit all the data to \textit{Geotab}'s system.
This allows \textit{Geotab} to gather live engine information, this information is then interpret and analysed according to some criterias which makes the system able to, among others, detect vehicle accidents remotely\footnote{Geotab remote accident tracking \url{https://docs.google.com/document/d/1rOJZzzZhoKXWrmswaseUr3Wk8h_nPXPK4yFtdi6yVJ0/edit}}.

Lastly \textit{Fleetio};
Whereas all the previous solutions were either fully automated or manual, \textit{Fleetio} takes advantage of the best of both worlds.

In the scenarios where the system was fully automated, the solutions required the company to invest in devices which should be mounted in each vehicle.
This is done in order for the solutions to gather then necessary information.
In the exact opposite case where the solution were completely manual this would require the company to invest a noticeable amount of time into the system in order for it to be up to date.

In order for \textit{Fleetio} to automate as much as possible, \textit{Fleetio} utilizes the phone of the driver to track the current position, speed of the vehicle and estimated fuel consumption.

For information such as calculating the expense of each of the drives by manually inputting the cost after refuelling each vehicle.
\textit{Fleetio} is cheaper than the other solutions we have looked at, easy to install alternative to bigger and more extensive fleet management systems, which already exists.
\textit{Fleetio} has managed to locate a gap in the market, this gap being the use of pre--existing hardware, e.g. the drivers phone.
This helps maintaining a low initial cost since no new hardware is required.
If a company decides to change from another company such as either \textit{TomTom} or \textit{Geotab}, \textit{Fleetio} is compatible with with the hardware those solutions require.

Some of the factors to take into consideration is the operating cost for each of the systems.
The solutions from \textit{TomTom} and \textit{Geotab} are located at top of the list.

The monthly cost for the \textit{TomTom} solution ranges from \$16 to \$25 per month per vehicle\footnote{Price range for TomTom Webfleet\\ \url{https://telematics.tomtom.com/en_gb/webfleet/fleet-management/vehicle-tracking/entry-level/}}.
There are two price levels; \$16 limiting the vehicle to the UK and \$25 for tracking beyond UK borders.
However if a vehicle is to leave United Kingdom the price will rise to the price of \$25, and \textit{TomTom} is limited to only most of the European countries.

The solution from \textit{GeoTab} costs around \$24 per month per vehicle, however this solution also requires the OBD device which costs \$99 per vehicle\footnote{GeoTab Price \url{http://www.anythinggps.com/index.php?route=product/product&product_id=51}}.

At the other end of the scale the solution from \textit{Fleetio} is located.
\textit{Fleetio} has two different packages to choose between, the pro plan at \$3 per month per vehicle and the advanced plan at \$5 per month per vehicle\footnote{Pricing for Fleetio \url{https://www.fleetio.com/pricing}}.
Neither \textit{Fleetio} or \textit{GeoTab} limits the area where the devices work, unlike \textit{TomTom}.

\subsection{Application}\label{sub:Application}
The rest of this section is going to focus upon the \textit{Fleetio} solution, this is done since we discovered that more information is available for \textit{Fleetio} than any of the other solutions.
In order for \textit{Fleetio} to gather data, a mobile application has to be installed on each driver's smart phone.
Initially, static misc data has to be manually submitted to the system, however \textit{Fleetio} support the import of vehicle details via the vehicle identification number also know as VIN, this feature can save some time.

\subsubsection{Features}\label{ssub:features}
Some of the tools that \textit{Fleetio} support are the following:
\begin{description}
    \item[Vehicle Management] \hfill
    \begin{description}
        \item[Vehicle Profiles] \hfill \\
        Through \textit{Fleetio}, a company is able to create vehicle profiles either manually or through import via VIN. This vehicle information can range from documents to images of the vehicles.
        It is then possible to search through the vehicles which are still in use, and disposed vehicles.
    \end{description}

    \item[Fleet Maintenance] \hfill
    \begin{description}
        \item[Scheduling] \hfill \\
        After all the information has been manually entered into \textit{Fleetio}, a company can then schedule a lot of different vehicle related tasks through the dashboard.
        These things can range from service reminders to preventative maintenance.
        \item[Issue Reporting] \hfill \\
        If a driver spots a problem or experiences any issue with a designated vehicle, the driver can then manually report those issues through the \textit{Fleetio} dashboard.
        After the report has been created an alert is send to the company to inform them of the issue.
        \item[Vendor Overview] \hfill \\
        \textit{Fleetio} makes it possible to map each expense to the different vehicle to the different vendors a company uses.
        \item[Maintenance History] \hfill \\
        Through the \textit{Fleetio} dashboard it is possible to keep track of all the repairs which have been performed on each of the vehicles through the maintenance history, this information must be manually plotted into the system after a repair is done.
    \end{description}

    \item[Driver Management] \hfill
    \begin{description}
        \item[Driver scheduling] \hfill \\
        \textit{Fleetio} supports manual scheduling of each driver and inspections of each vehicle.
        \item[Employee profiles] \hfill \\
        \textit{Fleetio} allows for a company to store information on each of the drivers and its other employees.
        \item[Real--time cost per kilometre] \hfill \\
        \textit{Fleetio} cooperates with multiple different fuel card partners.
        This makes it possible to calculate the cost per kilometre by combining the data from the fuel card partners and the tracking data from \textit{Fleetio}.
        If a fuel card from a non--cooperating company or an alternative payment method has been used, \textit{Fleetio} is still able to calculate the price, however this require the fuelling cost to be manually entered.
     \end{description}
\end{description}

\subsection{Analysis \& Synthesis}\label{sub:AandS}
Even though \textit{Fleetio} manages to be able to use the more advanced hardware such as \textit{Geotab}'s Plug-\&-Play OBD--II device, it does not take full advantage of it.
When \textit{Fleetio} uses pre--existing hardware, it takes gathered data and uses it to substantiate already existing data such as the current speed, which e.g. \ac{OBD} can measure more precisely.
\textit{Fleetio} also require manual input of several different data points, such as last refuelling.

In order to improve upon the pre--existing platform which \textit{Fleetio} has already developed, the support of third party modules such as \ac{OBD} modules, could improve the platform.
By looking at the already existing solutions the reuse of some functionality might be beneficial, some functionality could be the following:
\begin{description}
    \item[Vehicle profiles] \hfill \\
    The vehicle profiles are used to keep track of the documents and information which are associated with each vehicle in a company.
    \item[Maintenance history] \hfill \\
    The document of the maintenance history and which repairs that have been performed on each vehicle.
    \item[Employee profiles] \hfill \\
    An overview of each of the employees, and the ability to store specific information, e.g. the driving behaviour of a driver.
    \item[Real--time cost per kilometre] \hfill \\
    The tracking of the different vehicles position, and the amount of petrol in the tank allows for calculating the overall fuel usage, and if the price is given, an overall cost can be calculated.
\end{description}

However, some functionality might not apply to all companies within the scope of a similar system, some of these functionalities could be the following:
\begin{description}
    \item[Vendor overview] \hfill \\
    In a similar solution it might not be deemed as necessary to keep track of which vendors are used, since it is more important to keep track of the overall expense rather then the places where money is spend.
    \item[Driver scheduling] \hfill \\
    The scheduling of the drives is not relevant for this project, since the focus is on the tracking of vehicles and the properties of the vehicles.
\end{description}

\subsection{Assessment}\label{sub:Assessment}
In order to implement a fleet management system, a reliable back--end which will be able to support some of the features who was deemed beneficial for a similar system is important.
This leads us to focus on the missing aspects of the pre--existing solution \textit{Fleetio}.

%\fxnote{Indsæt noget rød tråd her, noget ref til næste sektion/kapitel og nævn hvad det skal bruges til. Er lidt usikker på hvor i rapporten denne sektion skal være, derfor jeg laver denne note og indsætter det når jeg har lidt bedre kontekst på det.}
