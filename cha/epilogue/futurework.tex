%vim: tw=80;spell
%Meta
%Meta Shit%
This chapter considers improvements that could be made to the existing system.
These improvements covers both areas we have not developed, and areas where we could improve upon the current solution.
%%%%%%%%%%%
\section{Producer Client}\fxnote{Is this enough? I can add more details but as this point i feel like we covered producer client pretty well}
%Data provider that handles data normalisation to replace sim
The current implementation only emulates a producer client.
This works for testing and demonstrating that the server works as intended, but the server also relies on the producer client.
We emulate data in a JSON format, which is the only format the server supports.
In \cref{sec:data_normalisation} we described that the ideal place for data normalisation would be on the producer client.
If we were to release the server without a producer client, any potential client would have to develop their own producer client, able to produce the data in a JSON format and normalise it, as well as a consumer client.

\bigskip
We envision the producer client as a smartphone app which utilises the phone's GPS and uses Bluetooth to communicate with an \ac{OBD} device.
The app then normalises the data from the two sources and uses the phone to transmit data to the server.
This is the most simple solution, as is described later in \cref{sec:fw_server} the client may be expanded to also allow for manual input or request data from other sources, in case such data would be relevant.
\section{Consumer Client}\fxnote{Same as producer, can expand on specific features but not sure if worthwhile}
%A baseline front--end client that allow for the simple functionality(fuel consumption, driving patterns, possible API extension to cooperation with fuel companies to determine price for a route, maintenance status.)
The current consumer client is a proof of concepts and still lacks a lot of features for it to be usable in a commercial sense.
While the development of a consumer client would be less important than that of a producer client, developing a baseline consumer client would not be unreasonable.
As described in \cref{sec:constituents_of_the_system} developing a consumer client that fits the need of any arbitrary user is not feasible.
Considering commercial use, smaller companies may not have the resources to develop their own consumer client, in that regard a baseline consumer client could suit some of their needs.
A baseline consumer client supporting features beyond what our proof of concept does, e.g. various statistics, would make our system usable for a wider clientele.
This requires that a producer client exists and as such the producer client would be the higher priority when it comes to future development.
\section{Improving the Server}\label{sec:fw_server}
Both the producer client and consumer client can be viewed as their own systems, and their development would as such not influence the server.
While the development of those two systems would contribute to a complete vehicle fleet management system, improvements to several areas of the server can also be made, some of which might influence the capabilities of a potential consumer client and the requirements for a producer client.
This section describes some of the possible improvements that could be made to the server, and also considers their influence on both producer and consumer clients.
\subsection{Server/Technological Changes}\fxnote{Jesper/SassyPants/Truls Plz}
Horizontal scale?
Back to NoSQL?
\subsection{More Endpoints}\fxnote{More endpoints is a very simple thing, not quite sure what to expand on here}
Delete Features
%More services??
\fxnote{enunciate liste} shows all the currently implemented endpoints.
The data obtainable through \ac{OBD} allows for the development of more endpoints as there are more details to be utilised, e.g. engine RPM.
The usefulness of this data varies from such things as engine RPM to accelerator pedal position.
A complete system would implement all the data available from \ac{OBD} however, this is a time--consuming process; ideally a prioritised list of all the data available would be made and used for continuous development on this.
\subsection{Stakeholder Specifics}
%In order to truly derive what future work can be done getting a stakeholder would be step one (i.e. some company that would use the system)
%Expand support to more vehicles, what does this truly mean, do we even have vehicle specific things at this moment? would it be to produce more services?
%Are we even specific at this moment, or would vehicle specific data be things to add?(Capacity for trucks perhaps or similar attributes)
%Danish military logistics division responded to email, expand with security focus in mind. This might require some hardware to handle security issues with OBD hardware.
%
%
At the current stage of the server there are no endpoints pertaining to a specific use case or type of fleet.
In \cref{sec:use_case} we decided to focus our development on lorry drivers, although this had no effect on the resulting system.
To further expand and support use case specific information, the majority of which would likely be manual input if not input from some other system.
Considering the various use cases would affect all the constituents of the vehicle fleet management system:
\begin{description}
    \item [The Producer Client] by allowing for manual input and perhaps acquiring information from other systems
    \item [The Server] by requiring more endpoints.
    \item [The Consumer Client] by using the information for business intelligence.
\end{description}
For lorry drivers an example may be such a thing as cargo weight which can not be acquired from \ac{OBD}, but could be input manually to the producer client, or perhaps acquired from some other existing systems.

As for catering to specific fields there are countless possible data points.
In order to find these data point, having a stakeholder in the given field to produce requirements for data distinct to their field is required.

In the end catering to specific fields would enable the production of more specialised consumer clients targeting the field in which the system is to be used.
For the server specifically as that is our focus, it would require a design that allows us to store data for a variety of fields without storing unwanted information, e.g. cargo weight for a taxi.

The functional scalability we have ensured throughout the development process also means that the addition of any field specific data would not be a difficult task.

\subsection{Long polling}
During the development we have been mostly focusing on services, and having the
minimal viable configuration to serve our consumer and provider. To expand on
the waypoint and route service, it would be useful for the consumer client to
have support for some event notification system. The easiest to implement, and
the only one with support in JAX-RS, is long polling. Long polling involves
having the server hold a connection open, until some event occurs, at that time
the server will return the event and close the connection. JAX-RS, and RESTEasy,
support long polling, or what they call ``asynchronous responses'', with the
\code{@Suspended} annotation. We did experiment with the feature, but found
a bug with the implementation which we have reported
upstream\footnote{\url{https://issues.jboss.org/browse/RESTEASY-1225?_sscc=t}}.
